%!TEX root = bioPrediction_main.tex
\section{Field Study}
We conducted an eight-week field study with 14 participants using experience sampling and biometric sensors to investigate how knowledge workers experience stress over time and the feasibility of predicting stress, focus, and awakeness based on biometric signals. 


\subsection{Participants}
We recruited 14 professionals via personal contacts from a large power and automation company. All participants work primarily in an office environment, though half of the participants spend at least 10\% (and up to 50\%) of their time in a laboratory environment. Office workers are a population that generalizes to a variety of contexts, and including part-time laboratory workers guarantees that our participants have varying work patterns that include different levels of computer usage, as well as different levels of activity in both individual and collaborative tasks.

Of the 14 participants, 11 are male and 3 are female. The average participant age is ~40, with 5 in the age range 25-34, 7 in the age range 35-44, and 2 in the age range 55+. The participants have an average number of years of professional experience of ~12, with 2 having less than 5 years, 10 having 5-15 years, and 2 having more than 25 years. All participants work for a research organization within the company, but their job functions span line management, laboratory science, scientific research, technology evaluation, and software development.


\subsection{Procedure}
We performed an in-situ study over the course of eight weeks. For this, we first informed participants about the study purpose and procedure, handed out biometric sensors and introduced and explained the self-reporting to the participants. For 10 of the participants, we further installed a computer activity tracker on their company-issued laptops (note: four participants from the original sample declined for privacy reasons). After the initial setup, participants were asked to fill out 3 surveys each day for the following eight weeks (see Section~\ref{sec:Surveys}). At the end of the eight weeks, we collected the biometric sensors and performed a short follow-up interview on the study and the participants' experiences.

In the second month of the study, we offered participants two one-hour Tai Chi classes per week (each in the middle of the day) in addition to the daily experience sampling and asked them to attend one class a week. We chose to offer Tai Chi class for two reasons: first, to incentivise participants for their continued participation and for keeping them engaged; and second, for their said benefits of reducing stress based on our consultation with researchers in psychiatry, psychology and more specifically on mindfulness practices. 



\begin{comment}
\subsection{Procedure}
\todo{This section does not work here. It is not really the study procedure, but rather the project procedure. Perhaps it can form the basis of an overview section after Field Study but before Analysis and Results.}

%\begin{figure}
%  \centering
%      \includegraphics[width=0.5\textwidth]{Figure1.png}
%  \caption{Process used to gather data and create the machine learning model able to predict the level of stress, focus, and awakeness} 
%  \label{process}
%\end{figure}
%\todo{The figure or text in the boxes could be shortened a lot, maybe even just replaced with pictures of sensor and survey like question with 5point scale. At this point, i'm not sure the figure adds a lot.}

%Figure \ref{process} shows the overall process performed in this study. We started by selecting a population
%of professionals in an office environment.

We use state-of-the-art sensors to capture a stream of biometric data from each individual and a computer interaction tracker to collect computer interaction information during the time
of the study. 

The participants were asked to fill a survey 3 times a day: two during the working day, and
one at the end of the day, where they rated their level of stress, focus, and awakeness during
the day; and their level of stress, awakeness, productiveness, and overall feeling at the end of the day.
This was conducted over an eight week period, which is
400\% longer than previous studies~\cite{zuger18,Muller16}.

We then pick a set of time windows (10sec, 20sec, 30sec, 45sec, 1min, 2min, 3min, 5min, 7.5min, 10min, 20min, 30min, 45min, 1hour, 2hour, 3hour) to analyze. We applied different statistical measurements (mean, standard deviation, variance, median, percentile25, percentile75, interquartile range, maximum, minimum, range) to each of the collected biometric signals and the computer interaction data within the time windows selected. We then 
compared the performance of four different machine learning 
algorithms to be able to predict the survey responses based in the biometric and 
computer interaction data.

Based in the initial corpus of training data, we create a machine learning model that is able to accurately predict the responses for stress, focus, and awakeness for each of the individuals wearing the sensors.
We are therefore able to infer the answer within the spectrum to each of these indicators without the need to ask the participants to provide an answer.
\end{comment}

\subsection{Data Collection}

In this section we describe the two datasets that we collected from each study participant.

\subsubsection{Biometric Sensors}
Figure~\ref{everion} illustrates Biovotion's Everion, which we used to track the biometric signals of the study participants. The Everion is worn on the upper arm and provides continuous monitoring of certain biometric measurements.\footnote{https://biovotion.zendesk.com/hc/en-us/categories/201633909-Everion-Device} Previous studies~\cite{zuger18,sano2013stress,healey2005detecting,wijsman2011towards,zuger2015interruptibility,goyal2017intelligent} have used similar devices~\cite{Okada11,polar,fitbitCharge} to capture psycho-physiological and biometric measurements for shorter periods of time or capturing a smaller number of measurements.

\begin{figure}
  \centering
      \includegraphics[width=0.4\textwidth]{Everion.jpg}
  \caption{We used Biovotion's Everion to collect biometric measurements from the participants}
   \label{everion}
\end{figure}

Table~\ref{signals} lists the biometrics measurements that we collected using the Everion. Each measurement is collected once per second, and each recorded observation has an associated timestamp and quality rating. Data collected by the Everion are uploaded to a server, from which we downloaded the data for use in our study.

\begin{table}[h!]
\begin{center}
\small\addtolength{\tabcolsep}{-1pt}
\begin{tabular}{l l}
\hline
Biometric Measurement & Units of Measure \\ 
\hline
\textbf{Physical Activity} & \cite{fox1999,aldana1996}\\
\hspace{3mm}Intensity of motion & (No unit)  \\
\hspace{3mm}Energy Expenditure & Calories per second (cal/s)  \\
\hspace{3mm}Step counter & Steps  \\
\hline
\textbf{Heart} & \cite{haapalainen2010psycho,healey2005detecting,mulder1992measurement,haag2004emotion} \\
\hspace{3mm}Heart rate & Beats per Minute (bpm) \\
\hspace{3mm}Blood pulse wave & (No Unit)  \\
\hspace{3mm}Heart rate variability (RMSSD) & Milliseconds (ms) \\
\hspace{3mm}Blood oxygenation  & Percent (\%)  \\
\hspace{3mm}Blood perfusion & (No unit)  \\
\hline
\textbf{Skin} & \cite{healey2005detecting,haag2004emotion} \\
\hspace{3mm}Galvanic skin response &  kOhm  \\
\hspace{3mm}Skin temperature &  Degrees Celsius ($^{\circ}$C)  \\
\hline
\textbf{Respiration} & \cite{mulder1992measurement,healey2005detecting,haag2004emotion,masaoka1997}\\
\hspace{3mm}Respiratory rate & Breaths per Minute (bpm)  \\
\hline
%\textbf{Environmental}\\
%\hspace{3mm}Barometric pressure & Milibar (mbar)  \\
%\hline
\end{tabular}
\caption{Biometric measurements captured by the Everion, organized by category and with references to previous works using similar data}
\label{signals}
\end{center}
%\vspace*{-4mm}
\end{table}

\subsubsection{Computer Interaction Data}
To gain a better understanding of our participants day-to-day work activities, we installed an open source computer interaction monitor (reference omitted for doubleblind). The monitor ran in the background on participants' computers and tracked the active windows, as well as the keyboard and mouse activity. Four of our participants opted out of this part of the study for privacy reasons (S6, S10, S12, and S13). Therefore, we only included the computer interaction data in our analysis for our more coarse-grained explanatory model but not in our analysis on the predictive model for stress in the moment. 

\subsubsection{Surveys}
\label{sec:Surveys}
Following guidelines from previous studies~\cite{Lalle16,Panwar18,Luo18} and 
following the preferrences of extensive user piloting, we sent via 
text message a survey request to each participant two times per 
work day. Pilot participants preferred these over other means, in part, due 
to them being accessible and noticeable anywhere in the office. 

We sent the 
first request at a random time between 9am and 11am 
and sent the second request at a random time between 1pm and 3pm. We 
randomized the request times to avoid either establishing or observing a 
standard behavioral pattern. That is, we did not want the participants to 
plan for the arrival of the survey request at a set time, and we did not 
want the survey request to overlap with a set daily behavior (e.g., coffee 
break every day at 2:30pm). Similarly, we avoided using tools which allow 
for too much freedom in response time~\cite{Adams18} since this would 
discurage participation in stressed timeframes and would bias the 
corpus. 
%This sentence was added in response to Reviewer 1 comment:
%There is already a lot of work on designing a new self-log, a self-report 
%tool which can blend into the workstation [1] or offers users more freedom 
%in time to self-report [2], or facilitate the self-log by using wearables.
The same survey was sent each time:
\begin{enumerate}
\item How awake are you right now?
\item How stressed do you feel right now?
\item How focused on work are you right now? 
\end{enumerate}
We used the phrase ``right now'' to capture each aspect in the moment (so as to permit later prediction of each aspect based on biometric data). The wordings of the questions are based on a previous survey of individuals in an organizational context~\cite{Gloor_etal:2010}. The use of awakeness (rather than sleepiness) in Question 1 is inspired by previous work~\cite{Wilhelm_Schoebi:2007} and to some extent also captures the ``arousal'' aspect of the affective space~\cite{Russell:1980}.


%\todo{we are not using the last survey for anything in this paper, so I'd leave it out}
One last survey was sent at the end of the day, at 4:25pm which asked the 
four different questions
detailed below:
\begin{enumerate}
\item How awake have you been today?
\item How stressed did you feel today?
\item How productive have you been today?
\item How do you feel about your work day?
\end{enumerate}

Following guidelines from similar previous studies~\cite{fogarty05,tanaka11}, we asked the participants to respond to each question using a 5-point Likert scale ranging from 1 (not at all awake/stressed/focused) to 5 (extremely awake/stressed/focused). Each participant response, as stored by Survey Gizmo, comprised the date, the time at which the response was initiated, the time at which the survey was submitted, the unique identifier for the participant, and the responses submitted by the participant.

%\subsection{Intervention}
%At the midpoint of our study, we introduced an intervention in the form of Tai Chi classes, with the goal of examining the efficacy of Tai Chi or other mindfulness excercises for reduction of office stress. Classes were offered two times a week in the middle of the work day. Participants were asked to attend one class a week. We recorded each participants attendance, including which of the two sessions they attended.