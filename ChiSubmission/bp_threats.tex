%!TEX root = bioPrediction_main.tex
\section{Threats to Validity}
There are numerous threats to validity to our study.

\textbf{External Validity:}
The results of our study may 
 not generalize to a broader population of office workers.
To mitigate this risk, we collected participants
from a wide variety of different departments
with different age ranges, genders, work experience, and 
working in different positions.

Another threat is that our results may notgeneralize
to a different office environment. We  conducted
this study in a typical office environment,  similar to many
among technology workers across the world.
These office environments control for a series of
variables to make them standard world wide such
as controlled temperature and lighting.

\textbf{Internal Validity:}
This study tries to find correlations between
biometric features and different productivity-related indicators (stress, focus, and awakeness).
Nonetheless, biometric signals are influenced by far more
variables than the ones this study comprehends.
Therefore, trying to draw a strong causality between the biometric
features and the productivity-related aspects would be inaccurate.
To mitigate this risk,  we collected the data
in a rote environment and in a regular manner 
to minimize the number of 
external causes that may affect each participant's
biometric signals.

It is possible that the amount of data collected
is not sufficienet
to draw valid conclusions. To address this threat, 
we collected a data for an eight week period, which is
400\% longer than previous studies~\cite{zuger18,Muller16}.


\textbf{Construct Validity:}
A threat to the study is that
there are other factors that might either influence the
human aspects of interest or that were considered but
are unrelated biometric signals.
To mitigate this risk, we used an state-of-the-art
device that captures a large number of highly accurate biometric
signals. We collected the most commonly analyzed
biometrics that historically have shown correlation with 
productivity aspects from each user.
In an effort to maintain this research applicable to real-world environments, we picked this already existing device, even when, as a trade-off, we could not capture more descriptive and more intrusive signals such as SDNN, SCL, SCR, eye tracking, or brain activity.

We also perform a grid analysis in our machine learning model
to pick the number of features that is more predictive
of the productivity-related indicators, 
therefore removing the biometric
features that may be unrelated to the indicators.











