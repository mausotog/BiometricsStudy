%!TEX root = bioPrediction_main.tex
\section{Discussion}

Humans experience stress, focus and awakeness in different ways.  In
this paper, we have attempted to study these mental states in the
workplace using both participant self-reports and biometric data. We
discuss implications from our study for the workplace, including ways
in which the information might inform digitally-controlled or
digitally-informed parts of the workplace. We also discuss 
challenges imposed by the data and possible future paths of
research.

\subsection{Impliciations for Workplaces}
Being able to accurately recognize periods of high-stress in knowledge
workers could enable more respectful workplaces. For example, an
ability to sense and predict stress in the moment could help companies
to prevent or de-escalate potentially dangerous situations in the
workplace, e.g. confrontation between co-workers. Building an 
understanding of stress and focus over time and in the moment
could also help create workplaces that are more conducive
to enabling knowledge workers to be more productive. For example,
this information could be used to feed an  
 an awareness dashboard of a team's stress level,
and avoid digital interruptions in high-stress or high-focus periods
similar to previous studies~\cite{zuger2017reducing}.  Building
an understanding of awakeness and focus could also enable the creation
of workplaces that are conducive to workers producing higher-quality
work. For example, if awakeness or focus wans, they might 
be enhanced by adapting lighting in the workplace 
or scheduling breaks to prevent focus loss.

Currently, the cost of biometric sensors and necessary infrastructure,
such as automated light and sound systems for adjusting the
environment, makes our approach most appropriate for high-value
workspaces, such as control rooms, command centers, or dispatch
offices. However, as standard office settings become more
personalizable (e.g., via adjustable desks, lighting, and sound
showers) and sensor costs decrease, our approach could be applied to
any office environment, and thus could impact a large percentage of
modern workers. As in modern cars, temperature and lighting could be
regulated on a per-person basis, which would allow the environment to
react to the person's current state and to maximize each person's
preferences and productivity (e.g., preferences of men and women in
temperature~\cite{Karjalainen07}). Further research would be needed to
identify how to balance needs across a group of office workers and how
to handle conflicting levels between different group members.



%%\vspace{0.05in}
%\noindent
%\textbf{Implications for HCI Design and Workspaces}
%% The results discussed in this paper establish great potential
%% to increase the well-being of knowledge workers and the quality of their work
%% through monitoring their biometric signals and computer interactions with the
%% goal of recognizing their stress, awakeness, and focus levels in an uninvasive 
%% and automatic manner. e

\subsection{Tai Chi Intervention}
As we described early, the second half of the field study data
collection involved an intervention of Tai Chi. This intervention was
used to keep participants motivated to continue over the long study
period.  To investigate whether the intervention had any effect, we
linked participants tai chi session attendance data with their
self-reported percieved stress levels. Building a linear mixed model
with tai chi attendance as a random effect, we found that tai chi
attendance contributed a small amount to decreased stress in the work
week immediately following the tai chi session (slope of -0.188, p <
0.05). However, we did not find a connection between attendance and
stress on the day of the session suggesting that tai chi might have
long-term effects, but is not conducive at relieving stress close
in time to the intervention.

%% %\vspace{0.05in}
%% \noindent
%% \textbf{Ground Truth and Self-Reporting:} 
\subsection{Ground Truth and Self-Reporting}
Studying mental states, such as stress, awakeness and focus,
requires collecting a valid ground truth from each participant.
We spent considerable time when designing the study
 determining the exact questions to ask of participants,
consulting experts in the area, and basing
the questions and wording on previous research and studies. 
Despite the care taken, it is possible that the 
gathered data lacks reliability and validity. Some have
questioned the
reliability and validity of self-reports as we used in our study
due to subjective biases, lack of care in reporting, and the
highly individual nature of reporting aspects such as
stress~\cite{Hernandez11,Hovsepian15}. In addition, in contexts such
as the workplace, as in our study,
participants  might be afraid to genuinely report levels
of aspects, such as sleepiness. Hence, there is a chance that the
self-reports we gatheered  do not adequately reflect the ground
truth of the underlying variables under investigation. It could even be
the case that certain biometrics might represent a more accurate
ground truth of the studied phenomena than the self-reports. This
suggests that a more confirmative study rather than an inquiry study
could be a better approach, and we will explore such routes in future
work.

%% %\vspace{0.05in}
%% \noindent
%% \textbf{Imbalanced Data:} 
\subsection{Imbalanced Data}
Study participants provided highly
imbalanced data in their survey responses, with most participants only
taking advantage of a subset of the Likert-scale values and the data
points mostly being clustered around the middle of the scale, as can
be seen in Table~\ref{responseDistribution}. While some of the
imbalance is expected due to certain classes, such as 'not stressed',
being more common in the workplace, this imbalance also provides
challenges in the training and assessment of a machine learning
classifier, as also found by others, e.g.~\cite{Exler16}. We addressed
this for the training by oversampling in case of few data samples for
the individual models and undersampling in case of a general model
where more data was available. Especially in light of this imbalance
in the data, the results we achieved with our models are
encouraging. For the assessment of the classifiers' performance we
addressed the imbalance by not just presenting accuracy, but also by
focusing on prediction and recall to examine the classifier's
performance in predicting the infrequent (yet more important) cases,
such as when a user is struggling to stay awake and an intervention or
warning might be needed most.
% \textbf{Unbalanced Data:}
% As shown in Table~\ref{responseDistribution}, the typical participant provided highly unbalanced survey responses, with most responses exhibiting a central tendency. Many participants provided zero or one very high (or very low) value on the Likert scale. In spite of this, our results show that by using biometric sensors and leveraging a machine learning approach, we are able to predict stress, focus, and awakeness better than the baseline, with improvements up to 84.64\%. Taking into account that the model correctly predicts many cases that happen infrequently, this is especially promising. These infrequent cases, such as when a user is highly stressed or struggling to stay awake, are in fact the most important cases to detect, as they are when intervention is needed most.

\subsection{Predicting Stress with Computer Interaction Data}
Knowledge workers, a focus of our work and study, often spend
a large amount of time each day interacting with information on 
their computer at work. An interesting direction for future study
is to consider whether this computer interaction data, which can
be gathered non-invasively as work occurs, could serve to sense
and predict stress, focus and awakeness. Features that could be
investigated include keystrokes per minute, mouse clicks per minute
and changes in the active title window.



%% %\vspace{0.05in}
%% \noindent
%% \textbf{Corpus Size:}
%% In this study we collected data from 14 professionals over the course of eight weeks. By scholarly standards this constitutes a large dataset --- especially given that the participants were not students. However, in the context of machine learning this corpus is relatively small. Thus, the model's overall performance, while promising, is only an indication of the performance that a larger corpus could provide. Our results show that there is a positive correlation between data added to the training corpus and the performance level of the model. This upwards trend likely continues, but a larger dataset is needed to confirm or reject this hypothesis. 

% \vspace{0.05in}
% \noindent
% \textbf{Data Collection:} 
% One advantage of our study is that it was conducted on professionals in a real office setting (i.e., not on students). However, the disadvantage to working with professionals is that business deadlines and other pressing real world issues lead to missing survey responses. Additionally, busy professionals often fall into the trap of always selecting the same value for a given question that they have answered many times, which reduces the accuracy of the collected data. To minimize the effect of this inertia, we advise collecting data at typical low-productivity times, such as in the early afternoon or on Mondays~\cite{mark2014bored}.

%% %\vspace{0.05in}
%% \noindent
%% \textbf{Per-User Model vs. Across-Users Model:}
%% Our results show that individually trained models outperform on average a more general model. Yet, given the increase in model performance with training data set size, we expect that with enough data, a generic model will perform reasonably close to the individually trained models. This would have significant practical implications, eliminating the need for survey-based feedback (which is only necessary for model training), and thereby completely automating measurement of human aspects. Our follow up work will focus on gathering more data toward this end.


% \vspace{0.05in}
% \noindent
% \textbf{Per-User Model vs. Across-Users Model:}\
% While we do see variations across participants, and thus our work shows that personal 
% models perform better than generic models, we expect that with enough data, a generic 
% model may perform equally well. This would have significant practical implications, eliminating the need for survey-based feedback (which is only necessary for model training), thereby completely automating measurement of human aspects. Our follow up work will focus on gathering more data toward this end and on evaluating the performance of a generic model.


