%!TEX root = bioPrediction_main.tex
\section{Conclusions} 
Stress, awakeness, and focus at work are highly relevant aspects when
it comes to productivity and well-being at the workplace. In this
paper, we presented the results of a study with 14 professional
knowledge workers in their workplace over an eight week period to
better understand how workers experience these human aspects over time and to
examine the ability of biometrics to predict these aspects. The longitudinal and in-situ placement of
the study support and extend previous work. Based on
daily collected survey responses, we observed that
although participants sometimes saw periods of sustained stress or sleepiness,
they would always return to their baseline reporting level
at some point. We also observed that stress levels seldom spiked,
but when they did rise, the rise in stress tended to last more
than a day. In addition to the survey responses, we 
continually collected
biometric data with which we were able to create a model that is able to
predict user stress, sleepiness, and lack of focus with small improvements in accuracy (from 3.1\% to 7.2\%, depending on the aspect in question), and moderate improvements in precision (25.9\% - 52.4\%) in comparison to a stratified random classifier. While the precision and recall scores we report are still low overall, this is a difficult problem to solve and our improvements indicate the potential for future researchers to build upon.

These results open up new opportunities to help increase knowledge
workers' productivity and well-being, ranging from instantaneously
taking action to prevent potentially risky situations and prevent
accidents due to a lack of focus or awakeness, all the way to
recommending interventions to reduce stress if it becomes more
chronic.

% \todo{as you can see, the conclusion is way too long and has some cut and paste repetition}

% Stress, awakeness and focus at work are highly relevant aspects when it comes to productivity and well-being at the workplace. In this paper, we presented the results of a study with 14 professional knowledge workers in their workplace over an eight week period to examine the ability of biometrics for predicting these producitivity-related aspects. The longitude and in situ placement of the study as well as the breadth of human aspects examined, support and extend previous work. Based on twice collected daily survey responses and continually collected biometrics data we were able to create a model that was able to predict , outperforming the baseline by as much as 84.6\%, in the case of awakeness. We also show that there is a positive correlation between number of training instances and model performance--a 114\% improvement from training with one sample to training with 33 samples--suggesting that with a larger amount of training data prediction would continue to improve. Due to this, while current user-specific models have higher performance than across-user models, there is hope that with enough data even across-user models could become accurate. 


% Our analysis shows that we are able to \todo{fill this}

% Our study with 14 professional knowledge workers in their workplaces over an eight week period has shown that we can use biometrics to predict these producitivity-related aspects of an individual knowledge worker. The longitude and in situ placement of the study as well as the set of human aspects examined support and extend previous work. In addition, our analysis 

% as well as computer interaction features to predict these aspects of an individual knowledge worker
% has shown that we can use biometrics as well as computer interaction features to predict these aspects of an individual knowledge worker over a long period of time in situ
% The results also show that the general ... \todo{fill this}

% In this study we analyze biometric signals of office workers to predict the productivity-related human aspects of stress, focus, and awakeness. A key component of this study was its length, an eight week period, longer than previous studies by a factor of four~\cite{zuger18,Muller16}, and its realism , studying professionals in situ instead of students. We collected information from 14 participants from different departments within the company, covering different ages ranges, genders, and work experience. By learning from twice daily survey responses and continually collected biometrics data we were able to create a model that was able to predict human aspect values, outperforming the baseline by as much as 84.6\%, in the case of awakeness. We also show that there is a positive correlation between number of training instances and model performance--a 114\% improvement from training with one sample to training with 33 samples--suggesting that with a larger amount of training data prediction would continue to improve. Due to this, while current user-specific models have higher performance than across-user models, there is hope that with enough data even across-user models could become accurate. 

% From these results emerge a series of opportunities to help increase awakeness, focus, and reduce stress in office workers. In the near future, we will therefore be able to detect a decrease in productivity-related aspects following our approach, and be able to take action to prevent potentially risky situations.

%\begin{acks}
%The acknowledgments section will be included for the camera-ready version of the paper.
%\end{acks}
