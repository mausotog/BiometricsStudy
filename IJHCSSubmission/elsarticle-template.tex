%\documentclass[review]{elsarticle}

%\documentclass[preprint,12pt,plainnat]{elsarticle}

%% Use the option review to obtain double line spacing
%% \documentclass[authoryear,preprint,review,12pt]{elsarticle}

%% Use the options 1p,twocolumn; 3p; 3p,twocolumn; 5p; or 5p,twocolumn
%% for a journal layout:
   \documentclass[final,1p,times,plainnat]{elsarticle}
%% \documentclass[final,1p,times,twocolumn,authoryear]{elsarticle}
%% \documentclass[final,3p,times,authoryear]{elsarticle}
%% \documentclass[final,3p,times,twocolumn,authoryear]{elsarticle}
%% \documentclass[final,5p,times,authoryear]{elsarticle}
%% \documentclass[final,5p,times,twocolumn,authoryear]{elsarticle}


%%
%% \BibTeX command to typeset BibTeX logo in the docs
%\AtBeginDocument{%
%  \providecommand\BibTeX{{%
%    \normalfont B\kern-0.5em{\scshape i\kern-0.25em b}\kern-0.8em\TeX}}}



\usepackage{lineno,hyperref}
\modulolinenumbers[5]

\usepackage{balance}       % to better equalize the last page
\usepackage{graphics}      % for EPS, load graphicx instead 
\usepackage[T1]{fontenc}   % for umlauts and other diaeresis
\usepackage{txfonts}
\usepackage{mathptmx}
%\usepackage[pdflang={en-US},pdftex]{hyperref}
\usepackage{color}
\usepackage{booktabs}
\usepackage{textcomp}
\PassOptionsToPackage{hyphens}{url}\usepackage{hyperref}

\renewcommand{\textrightarrow}{$\rightarrow$}

\usepackage{booktabs} % For formal tables
\usepackage{comment}
\usepackage{tabularx} 

% Some optional stuff you might like/need.
\usepackage{microtype}        % Improved Tracking and Kerning


\usepackage{todonotes}

\presetkeys{todonotes}{inline}{}
% for commenting and making it visible in the pdf (easy to hide after)
\newboolean{hidecomments}
\setboolean{hidecomments}{false}
%\nochangebars
\ifthenelse{\boolean{hidecomments}}
{\newcommand{\cb}[2]{}}
{\newcommand{\cb}[2]{
		\fbox{\bfseries\sffamily\scriptsize#1}
		{\sf\small$\blacktriangleright$ %$\RHD$
			{#2} $\blacktriangleleft$}}} %\LHD$}}}%
\newcommand\tf[1]{\cb{TF}{\textcolor{blue}{#1}}}
\newcommand\cs[1]{\cb{CS}{\textcolor{cyan}{#1}}}
\newcommand\ms[1]{\cb{MS}{\textcolor{red}{#1}}}
\newcommand\gm[1]{\cb{GCM}{\textcolor{magenta}{#1}}}
\newcommand\tfC[1]{\todo[color=green,inline]{Thomas: #1}}
\newcommand{\rev}[1]{\textcolor{blue}{#1}}

\journal{International Journal of Human-Computer Studies}

%%%%%%%%%%%%%%%%%%%%%%%
%% Elsevier bibliography styles
%%%%%%%%%%%%%%%%%%%%%%%
%% To change the style, put a % in front of the second line of the current style and
%% remove the % from the second line of the style you would like to use.
%%%%%%%%%%%%%%%%%%%%%%%

%% Numbered
%\bibliographystyle{model1-num-names}

%% Numbered without titles
%\bibliographystyle{model1a-num-names}

%% Harvard
%\bibliographystyle{model2-names.bst}\biboptions{authoryear}

%% Vancouver numbered
%\usepackage{numcompress}\bibliographystyle{model3-num-names}

%% Vancouver name/year
%\usepackage{numcompress}\bibliographystyle{model4-names}\biboptions{authoryear}

%% APA style
%\bibliographystyle{model5-names}\biboptions{authoryear}

%% AMA style
%\usepackage{numcompress}\bibliographystyle{model6-num-names}

%% `Elsevier LaTeX' style
\bibliographystyle{elsarticle-num}
%%%%%%%%%%%%%%%%%%%%%%%

\begin{document}

\begin{frontmatter}

%%
%% The "title" command has an optional parameter,
%% allowing the author to define a "short title" to be used in page headers.
\title{\rev{Observing and Predicting Knowledge Worker Stress, Focus and Awakeness in the Wild}}

%%
%% The "author" command and its associated commands are used to define
%% the authors and their affiliations.
%% Of note is the shared affiliation of the first two authors, and the
%% "authornote" and "authornotemark" commands
%% used to denote shared contribution to the research.
\author{Authors will be visible after double blind review}
%\vspace{110cm}
%\vspace{18ex}
%}


\begin{comment}
	
\author{Mauricio Soto}
\affiliation{%
\institution{ABB Future Labs}
\city{Raleigh}
\state{North Carolina}
}
\email{mauricio.soto@us.abb.com}

\author{Chris Satterfield}
\affiliation{%
\institution{The University of British Columbia}
\city{Vancouver}
\state{British Columbia}
}
\email{cds00@cs.ubc.ca}
	
\author{Thomas Fritz}
\affiliation{%
\institution{University of Zurich}
\city{Zurich}
\state{Switzerland}
}
\email{fritz@ifi.uzh.ch}
	
\author{Gail C. Murphy}
\affiliation{%
\institution{The University of British Columbia}
\city{Vancouver}
\state{British Columbia}
}
\email{murphy@cs.ubc.ca}
	
\author{David Shepherd}
\affiliation{%
\institution{Virginia Commonwealth University}
\city{Richmond}
\state{Virginia}
}
\email{shepherdd@vcu.edu}
	
\author{Nicholas Kraft}
\affiliation{%
\institution{UserVoice}
\city{Raleigh}
\state{North Carolina}
}
\email{kraft.nicholas.a@gmail.com} \\}
	
\end{comment}

\begin{abstract}
Knowledge workers face many challenges in the workplace: work is 
fragmented, disruptions are constant, tasks are complex, and work hours can be 
long. These challenges can impact knowledge workers' stress, focus and awakeness, and in turn their interaction with the digital environment, the quality of work 
performed and their productivity in general. We report on a field study with 14
knowledge workers over an eight-week period in which we investigated, using
experience sampling, how the workers experience stress, \rev{focus and awakeness} over time. During this field study, we also collected biometric data \rev{including heart- and skin-related measures}, which we then used to investigate if it is possible to predict stress, focus and awakeness, in the moment.
We observed and report on various trends in knowledge worker stress levels
over several weeks and show that machine learning models can be built
\rev{from the data of a single minimally invasive sensor}
to
predict stress, focus and awakeness \rev{accurately, with the abstract concept of focus
the hardest to predict.}
\end{abstract}

\begin{keyword}
Biometrics, \tf{Psycho-Physiological Data}, Stress, Awakeness, Focus, Computer Interaction, Ubiquitous Computing, Empirical Study, User Centered Design
\end{keyword}

\end{frontmatter}

%\linenumbers

%\maketitle

%!TEX root = bioPrediction_main.tex
\section{Introduction}

\vspace{-4mm}
\textit{`The most valuable asset of a 21st-century institution (whether 
business or non-business) will be its knowledge workers and their 
productivity'}~\cite{drucker1999knowledge}. Knowledge workers constantly face 
challenges, such as a high work fragmentation, continuous disruptions and 
distractions, highly complex and demanding tasks, and long working 
hours~\cite{gonzalez2004constant,mark2008cost,czerwinski04diary}. 
These challenges, amongst others, can lead to stress in the workplace.
Stress is an ever-growing concern as it can lead to
fatigue, burnout and various other illnesses, ultimately resulting in work 
absences and marked productivity 
losses~\cite{hockey1997stress,setz2010stress,wrs2010}.

Given the importance of understanding stress and its relationship and
effect on work, a number of studies have been conducted using a
variety of methods. Minimally invasive studies have shown the benefit of using
autonomic nervous system (ANS) activity by analyzing biometric signals
of the human body, such as blood pressure~\cite{kataoka00}.
These approaches can be used in long-term field
studies, opening up the question of what stress
looks like in the wild. 
In contrast, other studies have studied stress using invasive
techniques, such as measuring cortisol as a stress biomarker~\cite{piazza10}. These studies are best suited
for laboratory environments.

In this paper, we build on this ability to non-invasively monitor stress
to investigate two questions: RQ1) how do knowledge
workers at work experience stress over time, and RQ2) can we predict
whether a knowledge worker is experiencing stress in the moment based
on biometric data. A better understanding of how stress is experienced (RQ1)
can help inform the design of workplaces to manage
and alleviate stress. An ability to predict stress in the moment (RQ2)
can enable the development of digital tools to avoid stress. 

Our work builds upon previous work and extends it by performing an
eight-week study in the workplace with 14 knowledge 
workers. To the best of our knowledge, this is the 
longest in-situ study performed in a real-world 
environment of knowledge workers analyzing an extensive collection of biometric signals with experience samples examining real-time prediction of stress in the workplace~\cite{sano2013stress,healey2005detecting,wijsman2011towards,zuger2015interruptibility,goyal2017intelligent,Parnin2011,Nakagawa2014,Radevski2015}.
Previous studies are predominantly controlled lab studies which used only a subset of the relevant biometric signals we captured, much more limited in duration, or not focused on a real-world environment. 

We collected biometric data as well as
experience samples on stress, and two related human aspects: focus and
awakeness (i.e. wakefulness). We collected participant reports on focus because the
challenges that contribute to stress levels can also make it hard for
knowledge workers to stay focused and the level of focus
can affect
productivity~\cite{mark2014bored}.  We collected participant reports
on awakeness because a lack of awakeness (sleepiness) can result in
undesired consequences on work~\cite{connor02}.  The participants in
our study perform various job functions for a research and
development group within a single large corporation. For this study, they wore a novel, state-of-the-art
biometric armband sensor\footnote{Biovotion Everion
  sensor~\cite{everion}} with low invasiveness that captured heart-,
respiration- and skin-related measures. This modality was chosen to
ease longitudinal deployment in the field. Using machine learning, we
created classifiers and analyzed their ability to predict stress, focus, and awakeness levels. Considering all of these three human aspects helps us to consider whether one aspect might be easier to detect than another. If one aspect is easier to detect, it might serve
as a proxy or an indicator of the presence (or absence) of other aspects. 

In our analysis, we identify several trends in day-to-day stress levels
that emerged over the course of our eight-week study.  The results of
our analysis also show that biometric signals collected from a single
minimally invasive sensor can be used to predict stress and its
related aspects accurately, with the abstract concept of focus
(predictably) being the hardest to detect. Our results further show
that knowledge workers' self-reported levels of stress, focus and
awakeness and their physiological manifestation and prediction can
vary substantially between individuals.
%Our results further show that overall combining biometric features with computer interaction information 
%results in an improvement
%over each of the training sets individually.
The main contributions of our work are:
\begin{itemize}[topsep=0pt,itemsep=-1ex,partopsep=1ex,parsep=1ex]
	\item A qualitative examination of how knowledge worker stress presents and fluctuates in the wild based on an eight-week field study with 14 office workers.
	\item The creation and analysis of measures for the automatic monitoring of knowledge workers' stress, focus and awakeneess in the workplace based on a machine learning model trained on biometric data and experience samples.
	%\item A thorough analysis of ways to increase performance, including the training sample size, time window to collect data, and the inclusion of computer interaction features.
	\item A discussion on the impact of applying this research to improve
the interaction of knowledge workers with their digital environment leading to an improvement in their productivity and well-being, besides a reflection on aspects that can be further improved in future studies.
    \end{itemize}

%Ideally businesses believe that \textit{`The most valuable asset of a 21st-century institution (whether business or non-business) [is] its knowledge workers and their productivity'}~\cite{drucker1999knowledge}. In practice these valuable assets are as mercilessly optimized as assembly lines during the industrial age. Knowledge workers' desks occupy increasingly small footprints, optimizing a company's cost per square unit. Their shifts, especially in factories, command centers, or web service organizations, span twenty four hours a day, optimizing uptime. 
%
%
%
%In these settings they face a variety of challenges. Their squeezed workspaces lead to continuous disruptions and distractions, on top of the high work fragmentation and highly complex tasks typical of modern office work~\cite{gonzalez2004constant,zuger2015interruptibility,zuger18,mark2014bored}. %This leads to an up to 80\% of accidents being attributed, at least in part, to actions or omissions of workers\cite{hse99}. 
%Their shift work--and the required overnight shifts that accompany it--naturally causes sleepines. These challenges combine to increase stress and decrease focus, which in turn impacts worker productivity. For instance, the continuous disruptions that knowledge workers face can lead to higher error rates, lower overall performance~\cite{bailey2001effects,mark2008cost}, and, in a control room setting, could easily cause an accident that costs millions of dollars.
%%when a sleepy operator might, for example, miss early warning signs and cause...
%%\todo{Nick: can you find a good example here and citation?}. 
%Less immediate but just as insidious, stress at the workplace is a growing concern and one of the most common work-related health problems. It leads to fatigue, burnout and various other illnesses, ultimately resulting in work absences and marked productivity losses~\cite{hockey1997stress,setz2010stress,wrs2010}. 
%
%The first step in combating these common workplace challenges is to be able to measure their effect on workers. Yet measuring human aspects--focus, stress, and awakeness--has traditionally been done manually, making it difficult to efficiently apply Taylorism's scientific management. And yet the benefits of doing so are clear; measuring human aspects could provide better management of disruptions~\cite{iqbal2005effectiveness,zuger2017reducing}, automatically adjust lighting to circadian cycles to reduce sleepiness among shift workers, or continuously monitor stress levels to trigger pro-active interventions. 
%
%Fortunately, with recent advances in sensing technologies there is increasing hope that we can accurately and automatically measure human aspects. A number of recent studies have discussed and investigated the use of biometrics to measure aspects such as a person's cognitive and emotional states~\cite{beatty82,picard1995affective,Kramer90,Rowe98,richter1998psychophysiological,Wilson2002}. Specifically, studies have looked at using biometrics to measure stress~\cite{dishman2000stress,Setz2010}, or awakeness / (energetic) arousal~\cite{exler2016tired,mcduff2012affectaura,haag2004}. Yet, very few studies measure productivity-related factors in the workplace over an extended period of time, and little is known about measuring the more abstract concept of focus. Given the importance of focus, described as a combination of engagement and challenge in a work task by Mark et al.~\cite{mark2014bored}, to an office workers' productivity, this factor deserves further study. 
%
%
%Especially with the advances in sensing technologies, there is an increasing number of studies that discussed and investigated the use of biometrics to measure aspects such as a person's cognitive and emotional states~\cite{beatty82,picard1995affective,Kramer90,Rowe98,richter1998psychophysiological,Wilson2002}. Studies have also looked more specifically at using biometrics to measure stress~\cite{dishman2000stress,Setz2010}, or awakeness / (energetic) arousal~\cite{exler2016tired,mcduff2012affectaura,haag2004}. Yet, only very few studies have looked at measuring such productivity-related factors in the workplace over an extended period of time. Also, little is known about measuring a more abstract concept of focus that is often used in the context of knowledge workers' productivity and was, for example, described as a combination of engagement and challenge in a work task by Mark et al.~\cite{mark2014bored}. 

%Thus, the objective of our research is to continuously (and automatically) measure focus, awakeness, and stress in the workplace. These measures are targeted with an eye towards improving knowledge workers' productivity and well-being in the future. For instance, by automatically protecting the worker from audible interruptions during a high focus period by showering him/her with white noise or by recommending stress-reducing interventions such as Tai Chi during an extended period of high stress. 
%
%Our work builds upon previous work and extends it by performing an eight-week study in the workplace with 14 knowledge workers, collecting biometric and computer interaction data as well as experience samples on focus, awakeness, and stress. Our participants, who perform various job functions for a research and development group within a single large corporation, wore a single biometric armband sensor\footnote{Biovotion Everion sensor~\cite{everion}} with low invasiveness that captured heart- and skin-related measures. This modality was chosen to ease longitudinal deployment in the field. Using machine learning, we created classifiers and analyzed their ability to predict the three productivity-related human aspects. In addition, we compared biometric features with computer interaction features in their predictive power, we analyzed .. and ...
%
%The results of our analysis show that biometrics of a single minimally invasive sensor can be used to ... accurately, with the abstract concept of focus (predictably) being the hardest to detect, yet even its detection within reasonable limits was feasible. \tf{Chris: is it possible to state this?} The results also show that knowledge workers' self-reported levels of stress, focus and awakeness and their physiological manifestation and prediction can vary a lot between individuals. Our results further show that ...
%The main contributions of our work are the creation and analysis of measures for the automatic monitoring of knowledge workers' awakeness, stress, and the more abstract concept of focus in the workplace based on an eight-week field study with 14 office workers. a thorough analysis of ... and a discussion on the ...

%
%combine multiple factors, also including a more abstract concept of focus that is highly related to attention, but also engagement and challenge~\cite{mark2014bored}
%compare it with computer interaction sensors that are less invasive
%focus on a single biometric sensor modality to support longitudinal deployment and comfort

% Yet, only few studies looked at a more complete set of productivity-related factors
%focus, awakeness and stress predicting these productivity-related factors in combination, comparing biometrics to 
%I'd say it's the assessment of predicting multiple factors, using two different types of measurements (biometric and computer interaction), and doing all of it in practice over an extended period of time (every few are more than a few hours).

%stressed: high arousal, low valence
%tired: low arousal











%
%
%
%%\todo{NAK: I have heard from either Thomas or CGM a figure that captures how many problems are due to human error (something like 80\%). If we can find that cite, it should figure prominently in the motivation.}
%
%%Control rooms, while not much different from standard office settings on the surface, differ in the immediacy of their impact. In emergency dispatch offices, factory control rooms, and mining operations centers decision immediately impact safety and profits. A sleepy factory operator that misses early warnings could lose millions of dollars due to a process shutdown; a stressed mining operator that mistakenly dispatches two underground trucks to the same location could cause a deadly crash.  In these demanding settings  the environment must be finely tuned to support the job at hand. For instance, on overnight shifts, if controlling the lighting helps workers shift their circadian sleep cycle, reducing their sleepiness, it should be done, as the consequences of bad decisions are high.
%
%%To ensure the best decisions are made we focus on reducing stress, increasing focus and awakeness. Unfortunately, beyond a few lightly-studied phenomenon such as temperature's affect on awakeness, there are few absolutes that can be applied to workplaces wholesale. Thus, when optimizing a workspace tracking the affect of a given change, if any, is key. 
%
%%The challenge with tracking the affect of a given change in a workspace on its workers is that it is most often done via self-reporting, or the process of filling out daily, hourly, or worse surveys on one's current awakeness, stress, and focus. Worse, this process must be repeated every time a change is implemented. Performing this process rapidly becomes impractical since the changes in the environment may be very frequent  (e.g., modifying the room temperature several times a day). Previous studies show that interruptions that undesired constant interruptions that happen at inopportune moments have negative effects, causing more stress and frustration~\cite{adamczyk2004if,czerwinski2000instant,mark2008cost}.
%
%% The importance of the problem
%Human decisions in the workplace directly impact worker productivity, product/process quality, and workplace safety. Control rooms are noteworthy for the immediacy of impact. Control room operators make hundreds of decisions during a workday, and the consequences of poor decisions can be severe. A sleepy factory operator could cause millions of dollars in losses by failing to recognize the early warning signs of a process shutdown, an unfocused water-processing plant operator could cause a water line failure by missing an alarm warning of a pressure overload, or a stressed mining operator could cause a deadly crash by mistakenly dispatching two trucks to the same underground location.
%
%% The overall problem
%The goal of control room design is to create and maintain an environment that maximizes operator alertness (i.e., that decreases stress and that increases focus and awakeness). Modern control rooms rely on adjustments to ergonomics, light, and temperature to support improved operator alertness. For example, adjusting the lighting in a control room can help operators on overnight shifts to alter their circadian sleep cycles, improving their awakeness while at work. A key challenge in control room design is understanding the effects of a given change to the control room environment on the operators. In particular, reliance on operator self-reporting to capture current levels of stress, focus, and awakeness is cumbersome, distracting, and labor intensive. Indeed, self-reporting can affect the phenomena being measured; e.g., interrupting work to respond to a survey can reduce focus. Moreover, as self-reporting is required not only to create the initial environment, but also to maintain it over time, operator self-reporting is impractical and can be harmful to operator alertness. Previous studies establish that regular interruptions, particularly those that happen at inopportune moments, cause stress and frustration~\cite{adamczyk2004if,czerwinski2000instant,mark2008cost}.
%
%% What others have done
%Previous research addresses how various signals such as blood pressure, heartbeat, and temperature can be linked to psychological states and processes~\cite{Kramer90,Rowe98,Eekelen04,valentini10}. However, little research addresses how biometric signals can be used to predict proxies for alertness such as levels of stress, focus, and awakeness. Instead, previous research has focused on using biometric signals to assess task difficulty in software development~\cite{fritz2014using}, code quality online~\cite{Muller16}, or  interruptibility~\cite{zuger18}.
%
%% What we present in this paper
%\todo{NAK: Would like the opinion of Thomas here --- should we give more of our vision here or stick to the (current) approach of keeping it specific to this paper. That is, do we want to talk about using biometrics to monitor and improve the workplace, or just stick to talking about predicting based on biometrics?}
%The overall goal of our work is to create a machine learning 
%model based in anecdotal biometric
%data which can infer the level of stress, focus, and 
%awakeness of a person by analyzing their biometric data.
%This approach will remove the unnecessary constant interruptions by inferring
%the stress, focus, and awakeness of an individual from their biometric features.
%This will allows to assess if a change performed in the environment 
%of an individual is beneficial
%or prejudicial to their productivity and alertness and, therefore, be able to 
%act upon this information to reverse or increase the change. 
%
%% The contributions of this paper
%The contributions of this paper are:
%\begin{itemize}
%\item The results of an exploratory study on the viability of using biometric sensors to predict the stress, focus, and awakeness of an individual
%\item A machine learning approach to predict proxies for alertness using biometric signals
%\item Guidance regarding the optimal time window and learning curve to generate an effective predictive model
%\item A comparison between the proposed approach using biometric signals and an approach using computer interaction data
%\end{itemize}


%!TEX root = bioPrediction_main.tex
\section{Related Work}

%\gm{Move to section 2 to help address reviewers comments about relationship between factors}

The study and prediction approach used in this paper is related
to previous studies of \rev{stress, focus, and awakeness} and studies
using biometrics to predict them. We consider
related work in each of these categories in turn.

%% Previous approaches~\cite{zuger18,Lalle16,Panwar18} have shown the feasibility of applying machine learning 
%% to biometric data to measure variables such as stress, interruptibility, 
%% negative emotion, or confusion during computer 
%% interaction. These approaches vary in the 
%% variables examined, the invasiveness of the sensors, and the length and 
%% context of the studies. Many previous studies focused on 
%% shorter time frames (mostly hours, but in one case also up to five 
%% days~\cite{sano2013stress} 
%% or two weeks~\cite{zuger18}). To the best of the authors' knowledge, no 
%% other study has examined such an extensive period (eight weeks) in a real-
%% world office environment and the inspected variables (stress, awakeness, and focus).


\subsection{Stress}
Much previous work relates to identifying and mitigating stress in an office environment.
Previous studies have measured stress by taking one of two possible approaches.
The first approach is to measure plasma catecholamine and cortisol as stress biomarkers~\cite{piazza10}.
This approach is impractical for use over prolonged periods of time, as in our eight-week study.
Further, this approach is imprecise because of the delay from the stress stimulation to the stress response, which may take from minutes to hours~\cite{Chandola10,Hellhammer09}.

The second approach, which we have chosen for our study, is to measure autonomic nervous system (ANS) activity by analyzing biometric signals of the human body, such as blood pressure, heartbeat, and skin temperature~\cite{kataoka00,Eekelen04,valentini10}.
In particular, changes in heart rate variability are associated with cognitive and emotional stress~\cite{mcduff16,dishman2000stress}.
This second approach has been used successfully by several past studies~\cite{Force96,gal07,montano09}.

Zaman et al.~\cite{Zaman19} performed a study in which they measure stress and productivity in short sessions with 63 participants using biometric sensors (thermal facial camera, wrist EDA, chest breathing sensor, and facial camera). Different from this study, our corpus is gathered from real-world office workers. Also, our study spans for eight weeks allowing us to analyze the behavior or our participants through the weeks and providing a more extensive in-depth analysis of our participants over time.

Hovsepian et al.~\cite{Hovsepian15} have worked to obtain a standard for continuous stress assessment. They used sensors to conduct a seven-day lab study with 26 participants, as well as a field study with 20 participants. They found that their model showed significant improvement over simple heart rate variability measurements. However, this model requires the use of a suite of invasive sensors that would be impractical for a study the length of ours. 

Hernandez et al.~\cite{Hernandez14} investigated the use of a pressure-sensitive keyboard and a capacitive mouse as non-intrusive means for measuring computer users' stress levels. They found participants exhibited significantly increased typing pressure and mouse contact when in stressful conditions. McDuff et al.~\cite{mcduff16} experimented with a camera to measure photoplethysmographic signals indicative of cognitive stress. Vizer et al.~\cite{vizer_automated_2009} used keystroke and linguistic features to automatically measure stress levels in response to cognitive and physical stress conditions. All of these studies were performed in a controlled lab setting over a short duration and the results have yet to be replicated in the field.

Kocielnik et al.~\cite{kocielnik_smart_2013} developed a framework for unobtrusive and continuous measurement of stress in real life conditions. They equipped university staff members with a wristband sensor and combined this data with information from the participants' calendars over the course of four weeks. They observed that the data they collected reflected well the participants perceptions of their own stress levels. However, this work focused mostly on providing retrospective information to users so they can work to improve their own stress balance. They did not attempt to make observations about the big picture of participants stress profiles or make predictions in real time. Also using a wristband, Hernandez et al.~\cite{Hernandez11} studied the stress level during calls in a call center over a seven day period. They collected skin conductance measures and examined the interpersonal variability of reporting stress. 


Our study stands out from these works by nature of its length and focus on knowledge worker stress in everyday office life, using unobtrusive measures. To the best of our knowledge there is no longitudinal study which attempts to explain and predict knowledge worker stress that comes close to the length of our own study. The eight week duration gives us authority to speak on the nature of day-to-day fluctuations in knowledge worker stress levels.


\rev{Rather than trying to measure or predict stress, several studies have induced stress and examined its effect on work performance, motivation and others. For instance, Sarsenbayeva et al.~\cite{Sarsenbayeva19} induced stress and studied how it affects mobile interaction, showing that it can reduce completion time and accuracy during target acquisition tasks. Evans and Johnson~\cite{evans00} investigated the correlation between noise in the workplace and stress levels. They found that workers exposed to open-office noise showed aftereffects that indicate motivational deficits but found no difference in cortisol levels. The population for their experiment comprised 40 female clerical workers, who were randomly assigned to a control condition or to three-hour low-intensity noise room designed to simulate typical open-office noise levels.}

\rev{Stress in the workplace and its effects on service providers has also been analyzed in call centers~\cite{Hernandez11} and in the context of the perceived imbalance between resources and demands~\cite{cherniss80}. These studies considered several factors such as personality traits, career-related goals and attitudes, as well as life outside of work, and examined their correlation with stress levels and burnout.}

\rev{Finally, there is also some work that already used proprietary stress measures for other types of classification. For example, Mirjafari et al.~\cite{Mirjafari19} used a proprietary stress measure by Garmin together with other data collected through the sensing of mobile devices and built machine learning models to differentiate between low and high performing workers.}

%\tfC{this is very different}
%Mirjafari et al.~\cite{Mirjafari19} used mobile devices over a long period to gather data, including a proprietary stress level measure, and built machine learning models to group workers based on their workplace performance metrics. %Sarsenbayeva et al.~\cite{Sarsenbayeva19} induced stress and studied how it affects mobile interaction reducing completion time and accuracy during target acquisition tasks.


%Evans and Johnson~\cite{evans00} investigated the correlation between noise in the workplace and stress levels. They found that workers exposed to open-office noise showed aftereffects that indicate motivational deficits but found no difference in cortisol levels. The population for their experiment comprised 40 female clerical workers, who were randomly assigned to a control condition or to three-hour low-intensity noise room designed to simulate typical open-office noise levels.

%\tfC{following stuff needs to be rewritten, it's just wrong in the first place}
%When it comes to stress in the workplace, there have also been several studies to analyze its effects. For instance, 
%Stress in the workplace and its effects on service providers has been analyzed in call centers~\cite{Hernandez11} 
%and in the context of the perceived imbalance between resources and demands~\cite{cherniss80}. These studies considered several factors such as personality traits, career-related goals and attitudes, and life outside of work, and these factors were correlated with stress levels and burnout.






%Bader~\cite{Bader95} proposed an approach towards monitoring and estimating a person's awakeness by leveraging on a stationary pressure sensor to track the person's body movements and a correlator to produce a change signal which is compared with a preset wakefulness threshold in a threshold detector circuit.

%A wearable system for mood assessment considering smartphone features and data from mobile ECGs. Exler, Schankin, 
%Looked at energetic-arousal (tired-awake)
%This paper focuses on valence, and barely mentions energetic arousal 
%so I will leave it out

\subsection{Focus}
Focus refers to the allocation of limited cognitive processing resources~\cite{Anderson04}.
Mark et al.~\cite{mark2014bored} studied \rev{attentional states, including focus, for workplace activities by analyzing the digital activity of 32 information workers in situ for 5 days.} They found that boredom is highest in the early afternoon and focus peaks in the middle of the afternoon. They also found that doing work that requires focus correlates with stress, while rote work correlates with happiness.

Interruptions in the office are a common barriers keeping workers from sustaining focus on their work related activities, particularly when the interruptions occur at inopportune moments.
Such interruptions may include emails, alerts, or interactions with co-workers\cite{gonzalez2004constant,chong2006interruptions,shamsi07}. Interruptions in inopportune times can have negative effects that range from higher error rate and lower overall performance
to an increase in stress and frustration~\cite{bailey2001effects,czerwinski2000instant,mark2008cost}.
External interruptions may cause workers to enter a ``chain of distraction''~\cite{shamsi07}.
This chain is composed by stages of preparation, diversion, resumption and recovery that result in time away from an ongoing task. \rev{Since interruptions can have a large impact on the focus and productivity of office workers, several studies have examined the prediction of interruptibility---the availability for interruptions---using a variety of features, including computer interaction and biometrics~\cite{fogarty2005examining, iqbal2008effects, chen2004using, bailey2008understanding, zuger2015interruptibility, zuger18}. Most of these studies were again conducted for small and controlled tasks over shorter periods of time.}


Other studied constructs that relate to focus in the workplace include cognitive absorption, cognitive engagement, flow, and mindfulness.
Cognitive absorption describes periods of time in which a person experiences total immersion in an activity.
This state is also accompanied by a sense of deep enjoyment, a feeling of control, curiosity, and not realizing the passing of time.
It has been associated with ease of use and perceived usefulness of information technology~\cite{agarwal00}.
Cognitive engagement is described~\cite{webster97} as a period of strong focus in an activity without the feeling of a sense of control of the situation.
Flow~\cite{Csikszentmihalyi90}, and mindfulness~\cite{Weick06,dane11} are psychological states that describe periods of prolonged attention and total immersion in an activity. Flow occurs when a person is focused on an activity that requires high challenge and high use of the person's skills, whereas mindfulness is characterized by being aware of fine detail, affording the capacity to discover and manage unexpected events.

\subsection{Awakeness}

Sleepiness (lack of awakeness) and its associated risk of serious injury to passengers has been studied in the context of automobile 
accidents~\cite{connor02,Nordbakke07}.
These studies show a strong association between the level of acute driver sleepiness and the risk of injury crash.
Connor et al.~\cite{connor02} conducted a population-based case study using the Stanford sleepiness scale, which is similar to a seven-point Likert scale and describes seven different levels of sleepiness from ``Could not stay awake, sleep onset was imminent'' (1) to ``Felt active, wide awake'' (7).
Nordbakke and Sagberg~\cite{Nordbakke07} show that drivers are well aware of  various factors influencing the risk of falling asleep while driving. Drivers also have good knowledge of the most effective measures to prevent falling asleep at the wheel.
However, most of drivers continue driving even when recognizing sleepiness signals, due to the desire to arrive at a reasonable time, the length of the drive, or pre-planed commitments.


\subsection{Biometrics}
Our study investigates the prediction of stress, focus, and awakeness using two different types of measurements, and biometric signals over eight weeks in a real-life office setting.
Existing work~\cite{sano2013stress,healey2005detecting,wijsman2011towards,zuger2015interruptibility,goyal2017intelligent,Parnin2011,Nakagawa2014,Radevski2015}\rev{---some of it already mentioned above---}analyzes a broad array of biometric signals and correlates them with individual's cognitive states and processes.
For example, Zuger et al.~\cite{zuger18} used biometrics to sense interruptibility in the office~\cite{zuger18}.
%and found correlations between biometric signals and predicting code quality~\cite{Muller16}, to assess task difficulty in software development~\cite{fritz2014using}, to assess developer's mental load and perceived difficulty while working on small code snippets, and
Biometric signals have also been studied in the context of technology users. For example, Parnin~\cite{Parnin2011} analyzes electromyography to measure sub-vocal utterances, and how these might be correlated with the programmer's perceived difficulty of programming tasks. Similarly, biometrics have been used to measure code difficulty by using biometric sensors~\cite{fritz2014using}
and using Near Infrared Spectroscopy to measure developer's cerebral blood flow~\cite{Nakagawa2014}.

Eye tracking technology~\cite{Bednarik2006,Crosby1990,Rodeghero2014} and brain activity~\cite{Ikutani2014,Siegmund2014} 
have been used in previous studies to examine different tasks in an office environment.
Eye tracking has been used to analyze memory load and processing load by inspecting task-evoked pupillary response and pupil size~\cite{beatty82}. Similar studies have shown high correlation between pupil size and mental workload of subtasks~\cite{beatty82} and cognitive load~\cite{Klingner10}.
Brain activity has been associated with different mental states~\cite{Berger29} by analyzing specific frequency bands (alpha, beta, gamma, delta, and theta) using electroencephalography (EEG). The increase or decrease of some of these frequencies is correlated with
attentional demand and working memory load~\cite{Smith05,sterman93}.
In contrast to studies that use eye tracking or EEG, we focused on a less invasive technology that can be applied in a real world scenario.
Similarly, we considered adding additional metrics such as the Depression, Anxiety and Stress Scale (DASS) and the Perceived Stress Scale (PSS)~\cite{Ferdous15} but opted for the ones presented in the paper after several discussions with experts in the area, and after piloting surveys to maximize participant compliance.



%!TEX root = bioPrediction_main.tex
\section{Field Study}
We conducted an eight-week field study with 14 participants using experience sampling and biometric sensors to investigate how knowledge workers experience stress over time and the feasibility of predicting stress, focus, and awakeness based on biometric signals. 


\subsection{Participants}
We recruited 14 professionals via personal contacts from a large power and automation company. All participants work primarily in an office environment, though half of the participants spend at least 10\% (and up to 50\%) of their time in a laboratory environment. Office workers are a population that generalizes to a variety of contexts, and including part-time laboratory workers guarantees that our participants have varying work patterns that include different levels of computer usage, as well as different levels of activity in both individual and collaborative tasks.

Of the 14 participants, 11 are male and 3 are female. The average participant age is ~40, with 5 in the age range 25-34, 7 in the age range 35-44, and 2 in the age range 55+. The participants have an average number of years of professional experience of ~12, with 2 having less than 5 years, 10 having 5-15 years, and 2 having more than 25 years. All participants work for a research organization within the company, but their job functions span line management, laboratory science, scientific research, technology evaluation, and software development.


\subsection{Procedure}
We performed an in-situ study over the course of eight weeks. For this, we first informed participants about the study purpose and procedure, handed out biometric sensors and introduced and explained the self-reporting to the participants. For 10 of the participants, we further installed a computer activity tracker on their company-issued laptops (note: four participants from the original sample declined for privacy reasons). After the initial setup, participants were asked to fill out 3 surveys each day for the following eight weeks (see Section~\ref{sec:Surveys}). At the end of the eight weeks, we collected the biometric sensors and performed a short follow-up interview on the study and the participants' experiences.

In the second month of the study, we offered participants two one-hour Tai Chi classes per week (each in the middle of the day) in addition to the daily experience sampling and asked them to attend one class a week. We chose to offer Tai Chi class for two reasons: first, to incentivise participants for their continued participation and for keeping them engaged; and second, for their said benefits of reducing stress based on our consultation with researchers in psychiatry, psychology and more specifically on mindfulness practices. 



\begin{comment}
\subsection{Procedure}
\todo{This section does not work here. It is not really the study procedure, but rather the project procedure. Perhaps it can form the basis of an overview section after Field Study but before Analysis and Results.}

%\begin{figure}
%  \centering
%      \includegraphics[width=0.5\textwidth]{Figure1.png}
%  \caption{Process used to gather data and create the machine learning model able to predict the level of stress, focus, and awakeness} 
%  \label{process}
%\end{figure}
%\todo{The figure or text in the boxes could be shortened a lot, maybe even just replaced with pictures of sensor and survey like question with 5point scale. At this point, i'm not sure the figure adds a lot.}

%Figure \ref{process} shows the overall process performed in this study. We started by selecting a population
%of professionals in an office environment.

We use state-of-the-art sensors to capture a stream of biometric data from each individual and a computer interaction tracker to collect computer interaction information during the time
of the study. 

The participants were asked to fill a survey 3 times a day: two during the working day, and
one at the end of the day, where they rated their level of stress, focus, and awakeness during
the day; and their level of stress, awakeness, productiveness, and overall feeling at the end of the day.
This was conducted over an eight week period, which is
400\% longer than previous studies~\cite{zuger18,Muller16}.

We then pick a set of time windows (10sec, 20sec, 30sec, 45sec, 1min, 2min, 3min, 5min, 7.5min, 10min, 20min, 30min, 45min, 1hour, 2hour, 3hour) to analyze. We applied different statistical measurements (mean, standard deviation, variance, median, percentile25, percentile75, interquartile range, maximum, minimum, range) to each of the collected biometric signals and the computer interaction data within the time windows selected. We then 
compared the performance of four different machine learning 
algorithms to be able to predict the survey responses based in the biometric and 
computer interaction data.

Based in the initial corpus of training data, we create a machine learning model that is able to accurately predict the responses for stress, focus, and awakeness for each of the individuals wearing the sensors.
We are therefore able to infer the answer within the spectrum to each of these indicators without the need to ask the participants to provide an answer.
\end{comment}

\subsection{Data Collection}

In this section we describe the two datasets that we collected from each study participant.

\subsubsection{Biometric Sensors}
Figure~\ref{everion} illustrates Biovotion's Everion, which we used to track the biometric signals of the study participants. The Everion is worn on the upper arm and provides continuous monitoring of certain biometric measurements.\footnote{https://biovotion.zendesk.com/hc/en-us/categories/201633909-Everion-Device} Previous studies~\cite{zuger18,sano2013stress,healey2005detecting,wijsman2011towards,zuger2015interruptibility,goyal2017intelligent} have used similar devices~\cite{Okada11,polar,fitbitCharge} to capture psycho-physiological and biometric measurements for shorter periods of time or capturing a smaller number of measurements.

\begin{figure}
  \centering
      \includegraphics[width=0.4\textwidth]{Everion.jpg}
  \caption{We used Biovotion's Everion to collect biometric measurements from the participants}
   \label{everion}
\end{figure}

Table~\ref{signals} lists the biometrics measurements that we collected using the Everion. Each measurement is collected once per second, and each recorded observation has an associated timestamp and quality rating. Data collected by the Everion are uploaded to a server, from which we downloaded the data for use in our study.

\begin{table}[h!]
\begin{center}
\small\addtolength{\tabcolsep}{-1pt}
\begin{tabular}{l l}
\hline
Biometric Measurement & Units of Measure \\ 
\hline
\textbf{Physical Activity} & \cite{fox1999,aldana1996}\\
\hspace{3mm}Intensity of motion & (No unit)  \\
\hspace{3mm}Energy Expenditure & Calories per second (cal/s)  \\
\hspace{3mm}Step counter & Steps  \\
\hline
\textbf{Heart} & \cite{haapalainen2010psycho,healey2005detecting,mulder1992measurement,haag2004emotion} \\
\hspace{3mm}Heart rate & Beats per Minute (bpm) \\
\hspace{3mm}Blood pulse wave & (No Unit)  \\
\hspace{3mm}Heart rate variability (RMSSD) & Milliseconds (ms) \\
\hspace{3mm}Blood oxygenation  & Percent (\%)  \\
\hspace{3mm}Blood perfusion & (No unit)  \\
\hline
\textbf{Skin} & \cite{healey2005detecting,haag2004emotion} \\
\hspace{3mm}Galvanic skin response &  kOhm  \\
\hspace{3mm}Skin temperature &  Degrees Celsius ($^{\circ}$C)  \\
\hline
\textbf{Respiration} & \cite{mulder1992measurement,healey2005detecting,haag2004emotion,masaoka1997}\\
\hspace{3mm}Respiratory rate & Breaths per Minute (bpm)  \\
\hline
%\textbf{Environmental}\\
%\hspace{3mm}Barometric pressure & Milibar (mbar)  \\
%\hline
\end{tabular}
\caption{Biometric measurements captured by the Everion, organized by category and with references to previous works using similar data}
\label{signals}
\end{center}
%\vspace*{-4mm}
\end{table}

\subsubsection{Computer Interaction Data}
To gain a better understanding of our participants day-to-day work activities, we installed an open source computer interaction monitor (reference omitted for doubleblind). The monitor ran in the background on participants' computers and tracked the active windows, as well as the keyboard and mouse activity. Four of our participants opted out of this part of the study for privacy reasons (S6, S10, S12, and S13). Therefore, we only included the computer interaction data in our analysis for our more coarse-grained explanatory model but not in our analysis on the predictive model for stress in the moment. 

\subsubsection{Surveys}
\label{sec:Surveys}
Following guidelines from previous studies~\cite{Lalle16,Panwar18,Luo18} and 
following the preferrences of extensive user piloting, we sent via 
text message a survey request to each participant two times per 
work day. Pilot participants preferred these over other means, in part, due 
to them being accessible and noticeable anywhere in the office. 

We sent the 
first request at a random time between 9am and 11am 
and sent the second request at a random time between 1pm and 3pm. We 
randomized the request times to avoid either establishing or observing a 
standard behavioral pattern. That is, we did not want the participants to 
plan for the arrival of the survey request at a set time, and we did not 
want the survey request to overlap with a set daily behavior (e.g., coffee 
break every day at 2:30pm). Similarly, we avoided using tools which allow 
for too much freedom in response time~\cite{Adams18} since this would 
discurage participation in stressed timeframes and would bias the 
corpus. 
%This sentence was added in response to Reviewer 1 comment:
%There is already a lot of work on designing a new self-log, a self-report 
%tool which can blend into the workstation [1] or offers users more freedom 
%in time to self-report [2], or facilitate the self-log by using wearables.
The same survey was sent each time:
\begin{enumerate}
\item How awake are you right now?
\item How stressed do you feel right now?
\item How focused on work are you right now? 
\end{enumerate}
We used the phrase ``right now'' to capture each aspect in the moment (so as to permit later prediction of each aspect based on biometric data). The wordings of the questions are based on a previous survey of individuals in an organizational context~\cite{Gloor_etal:2010}. The use of awakeness (rather than sleepiness) in Question 1 is inspired by previous work~\cite{Wilhelm_Schoebi:2007} and to some extent also captures the ``arousal'' aspect of the affective space~\cite{Russell:1980}.


%\todo{we are not using the last survey for anything in this paper, so I'd leave it out}
One last survey was sent at the end of the day, at 4:25pm which asked the 
four different questions
detailed below:
\begin{enumerate}
\item How awake have you been today?
\item How stressed did you feel today?
\item How productive have you been today?
\item How do you feel about your work day?
\end{enumerate}

Following guidelines from similar previous studies~\cite{fogarty05,tanaka11}, we asked the participants to respond to each question using a 5-point Likert scale ranging from 1 (not at all awake/stressed/focused) to 5 (extremely awake/stressed/focused). Each participant response, as stored by Survey Gizmo, comprised the date, the time at which the response was initiated, the time at which the survey was submitted, the unique identifier for the participant, and the responses submitted by the participant.

%\subsection{Intervention}
%At the midpoint of our study, we introduced an intervention in the form of Tai Chi classes, with the goal of examining the efficacy of Tai Chi or other mindfulness excercises for reduction of office stress. Classes were offered two times a week in the middle of the work day. Participants were asked to attend one class a week. We recorded each participants attendance, including which of the two sessions they attended.
%%!TEX root = bioPrediction_main.tex

\subsection{Data Preparation}

In this section we describe how we preprocessed the collected data for use in training and testing machine learning models.

\subsubsection{Data Linking}

We linked the collected biometric data and survey responses for each participant. Linking the data is necessary to construct training and test datasets for use in creating and evaluation machine learning models.

Our linking approach is as follows. From the start time of each survey response, we look back one hour for available biometric data. For each minute in that hour-long time window, we check for biometric data to associate with the survey response. For example, if a participant started a survey response at 11:05am, we look for biometric data in the time frame 10:05am to 11:05am. If biometric data is available in the hour-long time window, we consider the survey response to have associated biometric data. Otherwise, we exclude the survey response from our dataset.

Reasons for a survey response to lack associated biometric data include:
\begin{itemize}
\item The participant not wearing the Everion in the hour before before beginning the survey
\item The Everion not recording data in the hour before the participant began the survey (e.g., due to low battery)
\item Data not being uploaded successfully to the server
%\item Compatibility issues between the sensor and the OS tracking the participant's data
%\item Issues accessing the data uploaded by the participants
\end{itemize}

Figure~\ref{surveyBio} illustrates the number of survey responses with associated biometric data for each study participant. Participant S2 and S12 have particularly low numbers of usable survey responses. In each of these cases, the issue related to biometric data not being uploaded successfully to the server.

\begin{figure}
  \centering
      \includegraphics[width=0.45\textwidth]{DuringTheDay.png}
  \caption{Availability of biometric data per participant. Green: survey responses with biometric
  data; red: responses with no biometric data.}
   \label{surveyBio}
   \vspace*{-6mm}
\end{figure}


\subsubsection{Feature Extraction}
We extracted features from the biometric data to provide as input to machine learning models. Previous studies~ \cite{vorburger05,zuger2015interruptibility} identify time windows as an important factor that impacts the prediction accuracy of a classifier. We considered many time windows from the literature on biometric analysis~\cite{zuger18}, ranging from 10 seconds to 3 hours. Specifically, we considered the following time windows: \textit{10sec, 20sec, 30sec, 45sec, 1min, 2min, 3min, 5min, 7.5min, 10min, 20min, 30min, 45min, 1hour, 2hour, 3hour}.

From the start time of each survey response, we look back the amount of time that corresponds to each time window, and we create features for all of the biometric data available in that time window. For example, if a participant started a survey response at 11:05am, for the 30min time window, we create features using all of the available biometric data from 10:35am to 11:05am. For each time window, we calculate 10 statistical measurements from the biometric data to create 10 distinct features. Specifically, the 10 statistical measurements are: mean, standard deviation, variance, median, percentile25, percentile75, interquartile range, maximum, minimum, and range. Thus, for each survey response, we generate a large number of corresponding features based on three factors: biometric measurement, time window, and statistical measurement. In addition to these biometric features, we also considered the time of day in which the questions were asked.

\subsubsection{Response Transformations}
Table~\ref{responseDistribution} illustrates the distribution of responses from each participant for each of the three survey questions (which are listed in Section~\ref{sec:Surveys}). The figure shows that there is a notable imbalance in the distribution of the self-reported responses provided by the participants. Most participants did not use all five points of the five-point Likert scale in their responses, and the distributions tend to skew toward one side or the other, depending on the question. Thus, we binarized the survey data into a two-point scale to give the machine learning models the best possible chance to make useful predictions. The two points in the binary scale represent negative or positive responses for each of the three human aspects of interests (e.g., not stressed or stressed). 

We binarized the survey responses as follows. For each participant, we calculated the median response value for each question. We classified each response below the median as 0 ('negative') and each response above the median as 1 ('positive'). The distribution for the stress question skewed left, so we included the median values in the 'positive' class, while the distributions for focus and awakeness skewed right, so we included those median values in the 'negative' class.

\subsubsection{Oversampling}
Even after binarizing the responses as described in the previous section, we found the distribution of responses was still quite imbalanced for many of our participants. This can be seen in the distribution columns in Table \ref{tab:accuracy}. To combat this, we applied random oversampling to our training sets, which artificially rebalances the dataset by creating randomly replicated data in the minority class. This has been a commonly used technique in previous studies on unbalanced datasets \cite{chawla2004,yap2014}.


\begin{table}
  \centering
      \includegraphics[width=0.5\textwidth]{distributiontable.pdf}
  \caption{Distribution of responses per participant over the 5-point Likert scale (1: not at all; 5: extremely) for each question. }
   \label{responseDistribution}
   \vspace*{-6mm}
\end{table}


%!TEX root = bioPrediction_main.tex
%%!TEX root = bioPrediction_main.tex
%%!TEX root = bioPrediction_main.tex
%\input{bp_results_1}
%%\input{bp_results_2}

\section{Analysis and Results}

\begin{table*}[h]
  \centering
  \includegraphics[width=1.0\textwidth]{rq1performance.pdf}
  \caption{Results of predictions using the individual models. The distribution columns show a bar chart of the response distribution (negative/positive) for each of the three variables. The baseline row represents the averaged results of our baesline classifier. The general row shows the averaged results of our models trained on all participants.}\label{tab:accuracy}%\todo{revise caption}}
  \vspace*{-4mm}
\end{table*}

To evaluate the efficacy of continuously predicting a knowledge worker's stress, focus, and awakeness in the workplace, we trained and tested machine learning classifiers using a leave-one-out cross validation. For prediction, we used the features extracted from the collected data as the input data, and binarized the participants' self-reported responses on stress, focus, and awakeness into two classes each (e.g., 'stressed' and 'not stressed') to use as output measures.

\subsection{Predicting Stress, Focus \& Awakeness}
% random forest best
\label{secOverallAccuracy}
As a first step, we compared multiple classifiers using the popular machine learning library scikit-learn~\cite{pedregosa11} and performing a grid search analysis to determine the optimal hyperparameters for each classifier. Our analysis showed that random forest outperforms all other classifiers, including Na\"ive Bayes, decision trees, support vector machine, random forest, and a multilayer perceptron neural network. The optimal values for random forest and the three output measures are listed in Table \ref{tab:hyperparams}. For the remainder of this paper, we will be referring to random forest classifiers trained with these hyperparameters.%\\[-0.1cm]
% for stress they were (minimum samples for split = 4, # of estimators = 100, max features = 0.5, k=300), for focus they were (minimum samples for split = 4, # of estimators = 50, max features = 0.5, k=200), and for awakeness they were (minimum samples for split = 4, # of estimators = 100, max features = 0.25, k=800)
\begin{table}[h]
	\begin{centering}
	\small\addtolength{\tabcolsep}{-1pt}
    \begin{tabular}{llll}
      \hline
      Variable & K & \# Estimators & Minimum Samples Split \\
      \hline
      Stress & 300 & 100 & 4\\
      Focus & 200 & 50 & 4\\
      Awakeness & 800 & 100 & 4\\
      \hline
    \end{tabular}
    \caption{The hyperparameters selected by grid search analysis to tune our random forest models. K refers to the number of features selected.}    \label{tab:hyperparams}
    \end{centering}
\end{table}



\noindent\textit{Results of Individual Classifiers}\\
Since peoples' experience of stress, focus, and awakeness (as well as their physiological manifestations) can vary a lot (e.g.,~\cite{Hernandez11}), we first trained individual classifiers for each participant (rather than a general one for all participants). The results of our analysis are reported in Table~\ref{tab:accuracy}. For our analysis, we report values of accuracy, one of the most commonly used metric to compare performance, as well as precision and recall of the classes of interest: 'stressed', 'not focused', and 'not awake'. Since the imbalance in the data can lead to high accuracy values if a classifier always just predicts the most likely/frequent class while ignoring the class of higher importance and interest, precision and recall of the class of interest can be more adequate metrics in this case~\cite{yap2014,bhattacharyya_data_2011,Hernandez11}. 
% However, the imbalance in the data can lead to a classifier with high accuracy if it always just predicts the more likely class while ignoring the class of higher importance and interest in our case (i.e. stressed, not focused, not awake). Therefore, we further report the precision and recall for the three classes 'stressed', 'not focused' and 'not awake'. 



Overall, we were able to use the extracted physiological features to predict all three aspects with reasonable accuracy, precision, and recall, as well as to improve upon the baseline---a stratified random classifier that randomly chooses one of the two classes with a bias towards the larger class. While the individually trained classifiers improved on average across all participants upon the baseline in all cases excepting recall of 'not focused', the improvement was substantially higher for awakeness (85\% improvement in precision,  62\% in recall, and 7\% in accuracy) than for stress or focus. Also, the performance of the individually trained classifiers varied greatly across participants. While some participants showed a large improvement, for others the baseline performed much better than the individually trained classifier. For instance, for predicting 'stressed', the individual classifiers improved upon the baseline for S4, S6, S8, S11, S12, and S14 with a maximum improvement of 88.4\% in precision and 184.6\% in recall for S12, while they did worse for S1, S3, S5, S7, S9, S10, and S13, and in the worst cases did not correctly predict a single instance of 'stressed'. Typically users that have the lowest precision and recall values are those where the data is the most unbalanced. This translates to a scenario where the classifier will likely label the minority instances as part of the majority class due to the unbalanced ratio of the data.\\[-0.1cm]
% We attribute these discrepancies to the large differences in response distributions between participants, as well as to the subjectivity of self-reporting.
%
%(improvement: 8\% precision, 1\% recall, 8\% accuracy), the individual performance varied greatly among participants. Some participants showed negligible difference in comparison to the baseline classifier, e.g. subject ..., while others showed a large improvement, e.g. subject ...
%\begin{table}[h!]
%\begin{centering}
%\begin{tabular}{lll}
%\hline
%Participant & Recall Improvement (\%) & Precision Improvement (\%)\\
%\hline
%S12 & 184.6 & 88.4 \\
%S6 & 66.7 & 85.8 \\
%S11 & 50.0 & 44.4 \\
%S8 & 32.4 & 26.7 \\
%S14 & 14.7 & 9.8 \\
%S4 & 5.6 & 5.5 \\
%S9 & -55.4 & -46.8 \\
%S3 & -90.0 & -82.1 \\
%S1 & -100.0 & -100.0 \\
%S5 & -100.0 & -100.0 \\
%S7 & -100.0 & -100.0 \\
%S10 & -100.0 & -100.0 \\
%S13 & -100.0 & -100.0\\
%S2 & - & -\\
%\hline
%\end{tabular}
%\caption{Percentage improvement in recall and precision for stress, using our approach compared to the baseline, on an individual level. For participant 2, the baseline achieved a precision and recall of 0, thus the improvement is undefined}
%\end{centering}
%\label{tab:indImprovement}
%\end{table}



\noindent\textit{Feature Selection and Importance}\\
There are a large variety of features that can be (and have been) calculated in previous research for each of the basic measurements listed in Table~\ref{signals}, such as the mean, standard deviation, maximum, and interquartile range. In addition, each of these metrics can be combined with the various time windows captured of a basic measurement, resulting in a large feature space. To reduce the feature space, we experimented with multiple feature selection methods, including selecting the top k highest correlated features by various metrics such as mutual information, Pearson's correlation coefficient, ANOVA's F-value, as well as wrapper methods such as recursive feature elimination, optimizing mean decrease accuracy by iteratively permuting features, and only selecting features that exceed a certain Gini importance threshold. We found that all methods produced similar results with respect to accuracy, precision, and recall for the individual models. Ultimately, we chose to use the top k features with the highest ANOVA F-value, as it is relatively simple and efficient to calculate. The values of k used were selected by grid search analysis, and are shown in Table \ref{tab:hyperparams}. 

%Respiration rate is also highly important for stress (#3 , 14.4%) but I'm not sure what previous works say about correlation with stress
Overall, the features that were selected as the important ones for the individual models based on the random forest algorithm varied greatly across participants. Yet, some feature categories were considered to be important more frequently than others. Table~\ref{tab:featureImportance} shows the averaged Gini importance for the feature categories used for predicting stress. Particularly important for stress were the feature categories heart rate variability (18.3\%) and skin temperature (15.7\%), both of which have been shown in several previous studies to be  indicators for stress~\cite{dishman2000stress,mcduff16,kataoka00}. For predicting focus, the feature categories for blood pulse wave (14.2\%) and heart rate variability (13\%) showed to be the most important categories, while for awakeness the most important ones were heart rate variability (13.6\%) and blood pulse wave (13.1\%).\\[-0.1cm]


\begin{table}[h!]
  \begin{centering}
  \begin{tabular}{llll}
    \hline
    Feature Category & Stress & Focus & Awakeness\\
    \hline
    Heart Rate Variability & 18.3\% & 13\% & 13.6\%\\
    Blood Pulse Wave & 10\% & 14.2\% & 13.1\%\\
    Heart Rate & 8.7\% & 12.6\% & 10.3\%\\
    Skin Temperature & 15.7\% & 9.8\% & 10\%\\
    Galvanic Skin Response & 6.6\% & 8.2\% & 5.1\%\\
    Respiration Rate & 14.8\% & 12.7\% & 10\%\\
    Oxygen Saturation & 5.6\% & 4.3\% & 2\%\\ 
    Energy Expenditure & 6\% & 7.7\% & 4.8\%\\
    Activity & 4.6\% & 7.8\% & 7.6\%\\
    Steps & 1.7\% & 0.8\% & 0.9\%\\
    Time of Day & 0\% & 0.1\% & 0.5\%\\
    \hline
  \end{tabular}
  \caption{The averaged Gini importance of each feature category, per response variable.}
  \label{tab:featureImportance}
  \end{centering}
  \vspace*{-2mm}
\end{table}


\noindent\textit{Individual vs.\ General Model}\\
Individual models are trained specifically for each individual and thus require a data collection period before they are capable of making accurate predictions. On the other hand, the idea of general models is to be able to train them on already collected data and then  to be able to apply them even to new and unseen individuals, thus overcoming the cold-start problem. Given the big individual differences in biometrics, training a general model to achieve an adequate accuracy for new individuals is not necessarily possible. 

To examine the performance of a general model for our participants, we trained three general models, one for focus, one for awakeness and one for stress. We roughly followed the same procedure as for the individual models. Due to the larger amount of data available in the general case, we used the more common random undersampling, which randomly selects elements in the majority class to exclude from the dataset, instead of random oversampling to balance the distribution of the dataset. The models were trained on the datasets of 13 of the 14 participants, and then evaluated on the dataset of the last, repeating this process for all 14 participants. 

The bottom row of Table \ref{tab:accuracy} presents the averaged performance results for this approach in terms of accuracy, precision, and recall. Although the averaged precision and recall are comparable or better than those of the averaged individual results, this was at the cost of a large decrease in overall accuracy. Upon closer investigation into the performance of the general model when testing on each participant, we found that individually trained models for each participant performed much better than a general model trained over all participants. Using stress as an example, for participant S12, for whom we saw the greatest increase compared to the baseline in individual models, the general model was unable to predict a single instance of 'stressed' correctly. This is consistent with our expectations because biometric features are highly specific to individuals.


%\todo{put the content of the following sentence somewhere into the discussion; We attribute these discrepancies to the large differences in response distributions between participants, as well as to the subjectivity of self-reporting.}


%%%!TEX root = bioPrediction_main.tex

\subsection{Minimum Number of Training Samples}\label{secLearningCurve}
Collecting experience samples from users is expensive, since participants 
are being interrupted several times a day and have to answer the survey. 
To minimize the number of samples to be collected from participants, we 
examined how the performance of individual classifiers changes over the 
number of samples used to train the classifier.

Participants in our study had varying levels of responsiveness to the 
experience sampling ranging from 10 to 76 (see Figure~
\ref{responseDistribution}).
%Answering Reviewer 2's concern: Why only 10 participants? What was the rationale for the cut-off selected? 
To answer this research question we analyze the maximum number of responses \textbf{all} analyzed participants have to train the model. Since some participants have a very small number of maximum answers (ten being the lowest) we excluded the four participants with the least number of responses for this analysis.
%To maintain a certain generalizability while 
%also being able to examine a range of sample numbers for training the 
%classifier, we removed the four participants with the least number of 
%samples for this analysis. 
As a result, we obtained a corpus of analysis where all participants have a 
higher number of responses (34 being the lowest), which allows us to examine 
the learning curve of the classifiers for a larger number of training data points. For our 
analysis, we thus performed a leave-one-out cross validation with random 
sample sets of size 1 to 33 and calculated the average through all folds of 
the validation.

For each of the three productivity-related aspects, we are again more 
interested in predicting when a worker is stressed, not awake, or not 
focused, the less common class in all three cases. Since the less common 
class can be very small, we weighted each participants' classifier 
performance by the percentage of the samples in this smaller class.


\begin{figure}
  \centering
          \includegraphics[width=0.5\textwidth]{20180912AwakenessLC2.png}
  \caption{Performance of the per participant trained awakeness classifiers, measured in precision, recall (sensitivity), and F-score. The dotted red line represents the F-score trend.}\label{fig:learningCurveInd}
  \vspace*{-3mm}
\end{figure}

\vspace{0.05in}
\noindent\textit{Individual Classifiers for Binary Prediction}\\
The averaged performance of all individually trained random forest classifiers for 'awake' with respect to the training sample size is presented in Figure~\ref{fig:learningCurveInd}. The trend indicates a positive correlation between the number of samples in the training set and the classifiers' F-score performance, with an overall improvement of  114\% (from 0.14 to 0.30) in the F-score between a training set of one sample to one with 33 samples. The trends for
the remaining indicators are 29\% for stress and no overall improvement for focus.\\[-0.1cm]

%\noindent\textit{}
%When contrasting
%models trained and tested on the 
%biometric data of each user individually, against
%a general model trained and tested with the data of all users. 
%We can conclude
%that for up to 33 training samples analyzed in this study,
%the former has better performance when predicting the
%responses of each user than when training 
%a model with the data of several different users. 
%This is partially due to the 
%variability across users and the subjective nature
%of the responses (e.g.,where slightly awake for one person
%may be not awake for another). 

\begin{figure}
  \centering
      \includegraphics[width=0.5\textwidth]{20180914Awakeness5PointScaleOnly15Lines.png}
  \caption{Performance of the per participant trained classifiers for predicting 5-point awakeness, measured in root mean squared error.}
   \label{fig:learningCurve5}
\end{figure}

\noindent\textit{Predicting Five Classes}\\
In a second step, we analyzed a more fine-grained prediction using the initial 5-point Likert scale responses rather than the binarized ones as output measure. Figure~\ref{fig:learningCurve5} depicts the performance of the individually trained classifiers in terms of the root mean squared error. The root mean squared error represents the distance of the predicted from the actual value, which provides a more nuanced measure of the performance in the fine-grained prediction case. The figure shows a similar trend as for the binarized prediction, in that the root mean square error averaged over all ten participants decreases with more samples (from 0.49 to 0.41 root mean square error) and thus the performance increases. At the same time, the figure also shows that the performance results for the fine-grained prediction, again, vary substantially across participants.

\subsection{Minimum Time Window}\label{secMinimumTW}
In general, the less biometric data is needed to accurately predict a certain outcome measure, the easier and faster the analysis and data collection. To examine the optimal and minimum time window for the prediction of stress, focus, and awakeness, we used 16 different time windows from 10 seconds to 3 hours as depicted in Figure~\ref{timeWindows}. For our analysis, we then trained individual classifiers for each of the 16 time windows, using only features that had a time window smaller or equal to the time window rather than using all combinations of $\{Biometric Measures\} X \{Statistical Metrics\} X \{Time Windows\}$. We again used random forest and a leave-one-out cross validation to train individual classifiers. Since the number of features used for the training changed with each time window, we did not apply our feature selection in this case, but used all features available. Finally, due to the imbalance in the data, we again weighted each participants' classifier performance by the number of instances in the smaller class to calculate the average.

\begin{figure}
  \centering
      \includegraphics[width=0.4\textwidth]{timeWindows.png}
  \caption{Time windows of biometric data collected prior to each survey response: 10sec, 20sec, 30sec, 45sec, 1min, 2min, 3min, 5min, 7.5min, 10min, 20min, 30min, 45min, 1hour, 2hour, and 3hour.}
   \label{timeWindows}
\end{figure}



\begin{figure}
  \centering
      \includegraphics[width=0.5\textwidth]{20180912AwakenessTWBars.png}
  \caption{Performance (F-score) of individual classifiers trained on the different time windows to predict `awake'.}
  \label{timeWindowsPandR}
  \vspace*{-3mm}
\end{figure}

Figure~\ref{timeWindowsPandR} shows how the F-score changes for predicting 
`awake' over the 16 different time windows. The figure shows an increasing 
trend in the F-score, i.e. the higher the number of included time windows, 
the higher the F-score. However, there is one exception, the time window of 
1200 seconds that achieves a performance close to the one for the time 
window 10,800 seconds (3 hours), at which point all features are included. 
Overall, our results thus show that while using all time windows up to 3 
hours performs best, and outperforms the feature set that is solely based on 
a 10 second time window by 28\% (from 0.18 to 0.24), the performance for a 
time window of 1200 seconds is a good trade-off for selecting a shorter time 
window while maintaining high performance.



%We also compared the performance of a model created per user against
%a model created across users.
%To calculate the performance of the model created per user,
%for each user we create a model using the features 
%related to each time window for the selected user only, 
%while in the model created with data across users,
%we trained the model with the data related to each
%particular time window of all users.
%our results show that the personalized model created
%for each user outperforms in every case
%the model created the with data across all users.

%This study
%shows that in general larger time windows predict stress,
%focus, and awakeness more precisely than 
%shorter time windows. We can also notice
%a very gentle upwards trend
%with a inflexion points at 45 seconds and 180.
%There is a peak of performance in 45 seconds, 
%time window that could be used if time is a scarce resource.
%The performance stabilizes around 180 time window
%which indicates that collecting data for a longer time
%period will not increase significantly the performance of the approach.
%The difference in performance between the 
%lowest performing and the highest performing time window is 
%less than a 10\% improvement. 


\subsection{Computer Interaction Data} \label{secCI}
Given our focus on knowledge workers (i.e., workers who generally spend a lot of time interacting with information on their computer at work), we also analyzed the use of computer interaction features to predict focus, awakeness and stress. To collect computer interaction data, we used an open source computer interaction monitor (reference omitted for double-blind.)
%the open source PersonalAnalytics project\footnote{https://github.com/sealuzh/PersonalAnalytics}  \cite{meyer18} 
to track participant's mouse and keyboard activity, as well as details about their active window. The specifics of the features tracked are listed in Table \ref{tracker}. The tracker was installed on the computers of 10 of the 14 participants, with participants S6, S10, S12, and S13 opting out of this part of the study due to privacy concerns.  Therefore, we limited this analysis to the 10 participants for whom we could calculate all features. 

\begin{table}
\begin{center}
\small\addtolength{\tabcolsep}{-1pt}
\begin{tabular}{l l}
\hline

Feature collected by tool & Description \\ 
\hline
Total keystrokes per min& Sum of all types of keystrokes \\ 
Normal keystrokes per min&F[h] Not backspace and navigation \\ 
Backspace keystrokes per min& Backspace keystrokes \\ 
Navigation keystrokes per min& Arrow key keystrokes \\ 
Total clicks per min& Sum of all click types \\ 
Other clicks per min& Not right and left clicks \\ 
Left clicks per min& Left clicks \\ 
Right clicks per min& Right clicks \\ 
Scrolled distance per min& Scrolled distance in pixels \\ 
Moved distance per min& Mouse movements in pixels \\ 
Activity switches per min& Browser window title changes \\ 
Category switches per min& Activity performed category \\ 
\hline
\end{tabular}
\caption{Features collected per user by the computer interaction tracker}%~\cite{meyer18}}
\label{tracker}
\end{center}
\vspace*{-1mm}
\end{table}

For calculating computer interaction features, we again used the aforementioned 16 time windows and scaled the computer interaction values if the time windows did not align. For our comparative analysis of the different sensing techniques---biometrics vs computer interaction---we then created two new feature sets for each participant in addition to the biometric one: one with only computer interaction features, and one with computer interaction features plus biometric features. 

Table~\ref{ciPerformance} lists the results of our analysis. The results show that in all cases, the computer interaction based model was able to improve upon the biometric model in terms of precision and recall.
%MS is removing the accuracy results because it is not what we care about and it is just making the results harder to understand
%, but not in accuracy. 
Further, we found that the combined model was the most effective model in terms of precision and recall for predicting stress and awakeness overall, but performed slightly worse than the model using only computer interaction features for focus. 
%Since we consider precision and recall for predicting the class of higher interest, i.e. stressed, not focused, not awake, to be the most important statistics when interpreting our results, we compared the models against each other by averaging these two statistics.


\begin{table}
\begin{center}
%\small\addtolength{\tabcolsep}{-1pt}
\begin{tabular}{llllll}
\hline
Model/Feature Set & Precision & Recall & F-Score \\ %& Accuracy\\
\hline
\textbf{Awakeness}\\
\hspace{3mm}Biometrics only  & 0.269 & 0.314 & 0.289 \\ %& 0.808\\
\hspace{3mm}C.I. only  & 0.425 & 0.362 & 0.391 \\ % & 0.758\\
\hspace{3mm}Biometrics + C.I. & 0.390 & 0.404 & 0.400 \\ % & 0.791 \\
\hline
\textbf{Stress}\\
\hspace{3mm}Biometrics only  & 0.270 &	0.260 & 0.265 \\ % & 0.775\\
\hspace{3mm}C.I. only & 0.290 & 0.272 & 0.281 \\ % & 0.698 \\
\hspace{3mm}Biometrics + C.I. & 0.317 & 0.286 & 0.301 \\ % & 0.712\\
\hline
\textbf{Focus}\\
\hspace{3mm}Biometrics only & 0.251 & 0.256 & 0.253 \\ % & 0.716\\
\hspace{3mm}C.I. only & 0.332 & 0.342 & 0.337 \\ % & 0.742\\
\hspace{3mm}Biometrics + C.I. & 0.340 & 0.316 & 0.328 \\ % & 0.745\\
\hline
\end{tabular}
\caption{Comparison of the performance of predicting stress, focus and awakeness using the 3 different feature sets for the 10 participants. The performance is calculated as the average of the performance of the individual classifiers. Computer Interactions is abbreviated as C.I. here for readability. Precision and recall refer to the prediction of the more important classes, i.e. `stressed', `not awake', `not focused'.}
\label{ciPerformance}
\end{center}
\vspace*{-3mm}
\end{table}

As with the biometric models, the individual performance of both the computer interaction only models and the combined models varied quite a bit between participants. Using stress as an example again, in the computer interaction models 5 of the 10 participants saw improvements compared to the baseline, with a maximum improvement of 128\% in precision, and 78\% in recall. In the combined model for stress, 5 of the 10 participants saw improvements compared to the baseline, with a maximum improvement of 117\% in precision and 95\% in recall. Neither model was capable of correctly predicting any instances of 'stressed' for participant S7.

Since the number of features changes  depending on which feature set is used, we adjusted the feature selection parameter for each of the computer interaction and combined computer interaction/biometric models. The values reported in this section were achieved using the optimal feature selection parameters we found, which are shown in Table \ref{ciFeatureSelection}.

\begin{table}
\begin{center}
\begin{tabular}{lc}
\hline
Model/Feature Set & Number of Features Selected\\
\hline
\textbf{Stress}\\
\hspace{3mm}C.I. & 400\\
\hspace{3mm}Biometrics + C.I. & 800\\
\hline
\textbf{Focus}\\
\hspace{3mm}C.I. & 20\\
\hspace{3mm}Biometrics + C.I. & 300\\
\hline
\textbf{Awakeness}\\
\hspace{3mm}C.I. & All\\
\hspace{3mm}Biometrics + C.I. & 50\\
\hline
\end{tabular}
\caption{The optimal number of features we found to select for each of the model/feature set combinations. Computer interactions is abbreviated as C.I.}
\label{ciFeatureSelection}
\end{center}
\vspace*{-4mm}
\end{table}



\section{Analysis and Results}

\begin{table*}[h]
  \centering
  \includegraphics[width=1.0\textwidth]{rq1performance.pdf}
  \caption{Results of predictions using the individual models. The distribution columns show a bar chart of the response distribution (negative/positive) for each of the three variables. The baseline row represents the averaged results of our baesline classifier. The general row shows the averaged results of our models trained on all participants.}\label{tab:accuracy}%\todo{revise caption}}
  \vspace*{-4mm}
\end{table*}

To evaluate the efficacy of continuously predicting a knowledge worker's stress, focus, and awakeness in the workplace, we trained and tested machine learning classifiers using a leave-one-out cross validation. For prediction, we used the features extracted from the collected data as the input data, and binarized the participants' self-reported responses on stress, focus, and awakeness into two classes each (e.g., 'stressed' and 'not stressed') to use as output measures.

\subsection{Predicting Stress, Focus \& Awakeness}
% random forest best
\label{secOverallAccuracy}
As a first step, we compared multiple classifiers using the popular machine learning library scikit-learn~\cite{pedregosa11} and performing a grid search analysis to determine the optimal hyperparameters for each classifier. Our analysis showed that random forest outperforms all other classifiers, including Na\"ive Bayes, decision trees, support vector machine, random forest, and a multilayer perceptron neural network. The optimal values for random forest and the three output measures are listed in Table \ref{tab:hyperparams}. For the remainder of this paper, we will be referring to random forest classifiers trained with these hyperparameters.%\\[-0.1cm]
% for stress they were (minimum samples for split = 4, # of estimators = 100, max features = 0.5, k=300), for focus they were (minimum samples for split = 4, # of estimators = 50, max features = 0.5, k=200), and for awakeness they were (minimum samples for split = 4, # of estimators = 100, max features = 0.25, k=800)
\begin{table}[h]
	\begin{centering}
	\small\addtolength{\tabcolsep}{-1pt}
    \begin{tabular}{llll}
      \hline
      Variable & K & \# Estimators & Minimum Samples Split \\
      \hline
      Stress & 300 & 100 & 4\\
      Focus & 200 & 50 & 4\\
      Awakeness & 800 & 100 & 4\\
      \hline
    \end{tabular}
    \caption{The hyperparameters selected by grid search analysis to tune our random forest models. K refers to the number of features selected.}    \label{tab:hyperparams}
    \end{centering}
\end{table}



\noindent\textit{Results of Individual Classifiers}\\
Since peoples' experience of stress, focus, and awakeness (as well as their physiological manifestations) can vary a lot (e.g.,~\cite{Hernandez11}), we first trained individual classifiers for each participant (rather than a general one for all participants). The results of our analysis are reported in Table~\ref{tab:accuracy}. For our analysis, we report values of accuracy, one of the most commonly used metric to compare performance, as well as precision and recall of the classes of interest: 'stressed', 'not focused', and 'not awake'. Since the imbalance in the data can lead to high accuracy values if a classifier always just predicts the most likely/frequent class while ignoring the class of higher importance and interest, precision and recall of the class of interest can be more adequate metrics in this case~\cite{yap2014,bhattacharyya_data_2011,Hernandez11}. 
% However, the imbalance in the data can lead to a classifier with high accuracy if it always just predicts the more likely class while ignoring the class of higher importance and interest in our case (i.e. stressed, not focused, not awake). Therefore, we further report the precision and recall for the three classes 'stressed', 'not focused' and 'not awake'. 



Overall, we were able to use the extracted physiological features to predict all three aspects with reasonable accuracy, precision, and recall, as well as to improve upon the baseline---a stratified random classifier that randomly chooses one of the two classes with a bias towards the larger class. While the individually trained classifiers improved on average across all participants upon the baseline in all cases excepting recall of 'not focused', the improvement was substantially higher for awakeness (85\% improvement in precision,  62\% in recall, and 7\% in accuracy) than for stress or focus. Also, the performance of the individually trained classifiers varied greatly across participants. While some participants showed a large improvement, for others the baseline performed much better than the individually trained classifier. For instance, for predicting 'stressed', the individual classifiers improved upon the baseline for S4, S6, S8, S11, S12, and S14 with a maximum improvement of 88.4\% in precision and 184.6\% in recall for S12, while they did worse for S1, S3, S5, S7, S9, S10, and S13, and in the worst cases did not correctly predict a single instance of 'stressed'. Typically users that have the lowest precision and recall values are those where the data is the most unbalanced. This translates to a scenario where the classifier will likely label the minority instances as part of the majority class due to the unbalanced ratio of the data.\\[-0.1cm]
% We attribute these discrepancies to the large differences in response distributions between participants, as well as to the subjectivity of self-reporting.
%
%(improvement: 8\% precision, 1\% recall, 8\% accuracy), the individual performance varied greatly among participants. Some participants showed negligible difference in comparison to the baseline classifier, e.g. subject ..., while others showed a large improvement, e.g. subject ...
%\begin{table}[h!]
%\begin{centering}
%\begin{tabular}{lll}
%\hline
%Participant & Recall Improvement (\%) & Precision Improvement (\%)\\
%\hline
%S12 & 184.6 & 88.4 \\
%S6 & 66.7 & 85.8 \\
%S11 & 50.0 & 44.4 \\
%S8 & 32.4 & 26.7 \\
%S14 & 14.7 & 9.8 \\
%S4 & 5.6 & 5.5 \\
%S9 & -55.4 & -46.8 \\
%S3 & -90.0 & -82.1 \\
%S1 & -100.0 & -100.0 \\
%S5 & -100.0 & -100.0 \\
%S7 & -100.0 & -100.0 \\
%S10 & -100.0 & -100.0 \\
%S13 & -100.0 & -100.0\\
%S2 & - & -\\
%\hline
%\end{tabular}
%\caption{Percentage improvement in recall and precision for stress, using our approach compared to the baseline, on an individual level. For participant 2, the baseline achieved a precision and recall of 0, thus the improvement is undefined}
%\end{centering}
%\label{tab:indImprovement}
%\end{table}



\noindent\textit{Feature Selection and Importance}\\
There are a large variety of features that can be (and have been) calculated in previous research for each of the basic measurements listed in Table~\ref{signals}, such as the mean, standard deviation, maximum, and interquartile range. In addition, each of these metrics can be combined with the various time windows captured of a basic measurement, resulting in a large feature space. To reduce the feature space, we experimented with multiple feature selection methods, including selecting the top k highest correlated features by various metrics such as mutual information, Pearson's correlation coefficient, ANOVA's F-value, as well as wrapper methods such as recursive feature elimination, optimizing mean decrease accuracy by iteratively permuting features, and only selecting features that exceed a certain Gini importance threshold. We found that all methods produced similar results with respect to accuracy, precision, and recall for the individual models. Ultimately, we chose to use the top k features with the highest ANOVA F-value, as it is relatively simple and efficient to calculate. The values of k used were selected by grid search analysis, and are shown in Table \ref{tab:hyperparams}. 

%Respiration rate is also highly important for stress (#3 , 14.4%) but I'm not sure what previous works say about correlation with stress
Overall, the features that were selected as the important ones for the individual models based on the random forest algorithm varied greatly across participants. Yet, some feature categories were considered to be important more frequently than others. Table~\ref{tab:featureImportance} shows the averaged Gini importance for the feature categories used for predicting stress. Particularly important for stress were the feature categories heart rate variability (18.3\%) and skin temperature (15.7\%), both of which have been shown in several previous studies to be  indicators for stress~\cite{dishman2000stress,mcduff16,kataoka00}. For predicting focus, the feature categories for blood pulse wave (14.2\%) and heart rate variability (13\%) showed to be the most important categories, while for awakeness the most important ones were heart rate variability (13.6\%) and blood pulse wave (13.1\%).\\[-0.1cm]


\begin{table}[h!]
  \begin{centering}
  \begin{tabular}{llll}
    \hline
    Feature Category & Stress & Focus & Awakeness\\
    \hline
    Heart Rate Variability & 18.3\% & 13\% & 13.6\%\\
    Blood Pulse Wave & 10\% & 14.2\% & 13.1\%\\
    Heart Rate & 8.7\% & 12.6\% & 10.3\%\\
    Skin Temperature & 15.7\% & 9.8\% & 10\%\\
    Galvanic Skin Response & 6.6\% & 8.2\% & 5.1\%\\
    Respiration Rate & 14.8\% & 12.7\% & 10\%\\
    Oxygen Saturation & 5.6\% & 4.3\% & 2\%\\ 
    Energy Expenditure & 6\% & 7.7\% & 4.8\%\\
    Activity & 4.6\% & 7.8\% & 7.6\%\\
    Steps & 1.7\% & 0.8\% & 0.9\%\\
    Time of Day & 0\% & 0.1\% & 0.5\%\\
    \hline
  \end{tabular}
  \caption{The averaged Gini importance of each feature category, per response variable.}
  \label{tab:featureImportance}
  \end{centering}
  \vspace*{-2mm}
\end{table}


\noindent\textit{Individual vs.\ General Model}\\
Individual models are trained specifically for each individual and thus require a data collection period before they are capable of making accurate predictions. On the other hand, the idea of general models is to be able to train them on already collected data and then  to be able to apply them even to new and unseen individuals, thus overcoming the cold-start problem. Given the big individual differences in biometrics, training a general model to achieve an adequate accuracy for new individuals is not necessarily possible. 

To examine the performance of a general model for our participants, we trained three general models, one for focus, one for awakeness and one for stress. We roughly followed the same procedure as for the individual models. Due to the larger amount of data available in the general case, we used the more common random undersampling, which randomly selects elements in the majority class to exclude from the dataset, instead of random oversampling to balance the distribution of the dataset. The models were trained on the datasets of 13 of the 14 participants, and then evaluated on the dataset of the last, repeating this process for all 14 participants. 

The bottom row of Table \ref{tab:accuracy} presents the averaged performance results for this approach in terms of accuracy, precision, and recall. Although the averaged precision and recall are comparable or better than those of the averaged individual results, this was at the cost of a large decrease in overall accuracy. Upon closer investigation into the performance of the general model when testing on each participant, we found that individually trained models for each participant performed much better than a general model trained over all participants. Using stress as an example, for participant S12, for whom we saw the greatest increase compared to the baseline in individual models, the general model was unable to predict a single instance of 'stressed' correctly. This is consistent with our expectations because biometric features are highly specific to individuals.


%\todo{put the content of the following sentence somewhere into the discussion; We attribute these discrepancies to the large differences in response distributions between participants, as well as to the subjectivity of self-reporting.}


%%%!TEX root = bioPrediction_main.tex

\subsection{Minimum Number of Training Samples}\label{secLearningCurve}
Collecting experience samples from users is expensive, since participants 
are being interrupted several times a day and have to answer the survey. 
To minimize the number of samples to be collected from participants, we 
examined how the performance of individual classifiers changes over the 
number of samples used to train the classifier.

Participants in our study had varying levels of responsiveness to the 
experience sampling ranging from 10 to 76 (see Figure~
\ref{responseDistribution}).
%Answering Reviewer 2's concern: Why only 10 participants? What was the rationale for the cut-off selected? 
To answer this research question we analyze the maximum number of responses \textbf{all} analyzed participants have to train the model. Since some participants have a very small number of maximum answers (ten being the lowest) we excluded the four participants with the least number of responses for this analysis.
%To maintain a certain generalizability while 
%also being able to examine a range of sample numbers for training the 
%classifier, we removed the four participants with the least number of 
%samples for this analysis. 
As a result, we obtained a corpus of analysis where all participants have a 
higher number of responses (34 being the lowest), which allows us to examine 
the learning curve of the classifiers for a larger number of training data points. For our 
analysis, we thus performed a leave-one-out cross validation with random 
sample sets of size 1 to 33 and calculated the average through all folds of 
the validation.

For each of the three productivity-related aspects, we are again more 
interested in predicting when a worker is stressed, not awake, or not 
focused, the less common class in all three cases. Since the less common 
class can be very small, we weighted each participants' classifier 
performance by the percentage of the samples in this smaller class.


\begin{figure}
  \centering
          \includegraphics[width=0.5\textwidth]{20180912AwakenessLC2.png}
  \caption{Performance of the per participant trained awakeness classifiers, measured in precision, recall (sensitivity), and F-score. The dotted red line represents the F-score trend.}\label{fig:learningCurveInd}
  \vspace*{-3mm}
\end{figure}

\vspace{0.05in}
\noindent\textit{Individual Classifiers for Binary Prediction}\\
The averaged performance of all individually trained random forest classifiers for 'awake' with respect to the training sample size is presented in Figure~\ref{fig:learningCurveInd}. The trend indicates a positive correlation between the number of samples in the training set and the classifiers' F-score performance, with an overall improvement of  114\% (from 0.14 to 0.30) in the F-score between a training set of one sample to one with 33 samples. The trends for
the remaining indicators are 29\% for stress and no overall improvement for focus.\\[-0.1cm]

%\noindent\textit{}
%When contrasting
%models trained and tested on the 
%biometric data of each user individually, against
%a general model trained and tested with the data of all users. 
%We can conclude
%that for up to 33 training samples analyzed in this study,
%the former has better performance when predicting the
%responses of each user than when training 
%a model with the data of several different users. 
%This is partially due to the 
%variability across users and the subjective nature
%of the responses (e.g.,where slightly awake for one person
%may be not awake for another). 

\begin{figure}
  \centering
      \includegraphics[width=0.5\textwidth]{20180914Awakeness5PointScaleOnly15Lines.png}
  \caption{Performance of the per participant trained classifiers for predicting 5-point awakeness, measured in root mean squared error.}
   \label{fig:learningCurve5}
\end{figure}

\noindent\textit{Predicting Five Classes}\\
In a second step, we analyzed a more fine-grained prediction using the initial 5-point Likert scale responses rather than the binarized ones as output measure. Figure~\ref{fig:learningCurve5} depicts the performance of the individually trained classifiers in terms of the root mean squared error. The root mean squared error represents the distance of the predicted from the actual value, which provides a more nuanced measure of the performance in the fine-grained prediction case. The figure shows a similar trend as for the binarized prediction, in that the root mean square error averaged over all ten participants decreases with more samples (from 0.49 to 0.41 root mean square error) and thus the performance increases. At the same time, the figure also shows that the performance results for the fine-grained prediction, again, vary substantially across participants.

\subsection{Minimum Time Window}\label{secMinimumTW}
In general, the less biometric data is needed to accurately predict a certain outcome measure, the easier and faster the analysis and data collection. To examine the optimal and minimum time window for the prediction of stress, focus, and awakeness, we used 16 different time windows from 10 seconds to 3 hours as depicted in Figure~\ref{timeWindows}. For our analysis, we then trained individual classifiers for each of the 16 time windows, using only features that had a time window smaller or equal to the time window rather than using all combinations of $\{Biometric Measures\} X \{Statistical Metrics\} X \{Time Windows\}$. We again used random forest and a leave-one-out cross validation to train individual classifiers. Since the number of features used for the training changed with each time window, we did not apply our feature selection in this case, but used all features available. Finally, due to the imbalance in the data, we again weighted each participants' classifier performance by the number of instances in the smaller class to calculate the average.

\begin{figure}
  \centering
      \includegraphics[width=0.4\textwidth]{timeWindows.png}
  \caption{Time windows of biometric data collected prior to each survey response: 10sec, 20sec, 30sec, 45sec, 1min, 2min, 3min, 5min, 7.5min, 10min, 20min, 30min, 45min, 1hour, 2hour, and 3hour.}
   \label{timeWindows}
\end{figure}



\begin{figure}
  \centering
      \includegraphics[width=0.5\textwidth]{20180912AwakenessTWBars.png}
  \caption{Performance (F-score) of individual classifiers trained on the different time windows to predict `awake'.}
  \label{timeWindowsPandR}
  \vspace*{-3mm}
\end{figure}

Figure~\ref{timeWindowsPandR} shows how the F-score changes for predicting 
`awake' over the 16 different time windows. The figure shows an increasing 
trend in the F-score, i.e. the higher the number of included time windows, 
the higher the F-score. However, there is one exception, the time window of 
1200 seconds that achieves a performance close to the one for the time 
window 10,800 seconds (3 hours), at which point all features are included. 
Overall, our results thus show that while using all time windows up to 3 
hours performs best, and outperforms the feature set that is solely based on 
a 10 second time window by 28\% (from 0.18 to 0.24), the performance for a 
time window of 1200 seconds is a good trade-off for selecting a shorter time 
window while maintaining high performance.



%We also compared the performance of a model created per user against
%a model created across users.
%To calculate the performance of the model created per user,
%for each user we create a model using the features 
%related to each time window for the selected user only, 
%while in the model created with data across users,
%we trained the model with the data related to each
%particular time window of all users.
%our results show that the personalized model created
%for each user outperforms in every case
%the model created the with data across all users.

%This study
%shows that in general larger time windows predict stress,
%focus, and awakeness more precisely than 
%shorter time windows. We can also notice
%a very gentle upwards trend
%with a inflexion points at 45 seconds and 180.
%There is a peak of performance in 45 seconds, 
%time window that could be used if time is a scarce resource.
%The performance stabilizes around 180 time window
%which indicates that collecting data for a longer time
%period will not increase significantly the performance of the approach.
%The difference in performance between the 
%lowest performing and the highest performing time window is 
%less than a 10\% improvement. 


\subsection{Computer Interaction Data} \label{secCI}
Given our focus on knowledge workers (i.e., workers who generally spend a lot of time interacting with information on their computer at work), we also analyzed the use of computer interaction features to predict focus, awakeness and stress. To collect computer interaction data, we used an open source computer interaction monitor (reference omitted for double-blind.)
%the open source PersonalAnalytics project\footnote{https://github.com/sealuzh/PersonalAnalytics}  \cite{meyer18} 
to track participant's mouse and keyboard activity, as well as details about their active window. The specifics of the features tracked are listed in Table \ref{tracker}. The tracker was installed on the computers of 10 of the 14 participants, with participants S6, S10, S12, and S13 opting out of this part of the study due to privacy concerns.  Therefore, we limited this analysis to the 10 participants for whom we could calculate all features. 

\begin{table}
\begin{center}
\small\addtolength{\tabcolsep}{-1pt}
\begin{tabular}{l l}
\hline

Feature collected by tool & Description \\ 
\hline
Total keystrokes per min& Sum of all types of keystrokes \\ 
Normal keystrokes per min&F[h] Not backspace and navigation \\ 
Backspace keystrokes per min& Backspace keystrokes \\ 
Navigation keystrokes per min& Arrow key keystrokes \\ 
Total clicks per min& Sum of all click types \\ 
Other clicks per min& Not right and left clicks \\ 
Left clicks per min& Left clicks \\ 
Right clicks per min& Right clicks \\ 
Scrolled distance per min& Scrolled distance in pixels \\ 
Moved distance per min& Mouse movements in pixels \\ 
Activity switches per min& Browser window title changes \\ 
Category switches per min& Activity performed category \\ 
\hline
\end{tabular}
\caption{Features collected per user by the computer interaction tracker}%~\cite{meyer18}}
\label{tracker}
\end{center}
\vspace*{-1mm}
\end{table}

For calculating computer interaction features, we again used the aforementioned 16 time windows and scaled the computer interaction values if the time windows did not align. For our comparative analysis of the different sensing techniques---biometrics vs computer interaction---we then created two new feature sets for each participant in addition to the biometric one: one with only computer interaction features, and one with computer interaction features plus biometric features. 

Table~\ref{ciPerformance} lists the results of our analysis. The results show that in all cases, the computer interaction based model was able to improve upon the biometric model in terms of precision and recall.
%MS is removing the accuracy results because it is not what we care about and it is just making the results harder to understand
%, but not in accuracy. 
Further, we found that the combined model was the most effective model in terms of precision and recall for predicting stress and awakeness overall, but performed slightly worse than the model using only computer interaction features for focus. 
%Since we consider precision and recall for predicting the class of higher interest, i.e. stressed, not focused, not awake, to be the most important statistics when interpreting our results, we compared the models against each other by averaging these two statistics.


\begin{table}
\begin{center}
%\small\addtolength{\tabcolsep}{-1pt}
\begin{tabular}{llllll}
\hline
Model/Feature Set & Precision & Recall & F-Score \\ %& Accuracy\\
\hline
\textbf{Awakeness}\\
\hspace{3mm}Biometrics only  & 0.269 & 0.314 & 0.289 \\ %& 0.808\\
\hspace{3mm}C.I. only  & 0.425 & 0.362 & 0.391 \\ % & 0.758\\
\hspace{3mm}Biometrics + C.I. & 0.390 & 0.404 & 0.400 \\ % & 0.791 \\
\hline
\textbf{Stress}\\
\hspace{3mm}Biometrics only  & 0.270 &	0.260 & 0.265 \\ % & 0.775\\
\hspace{3mm}C.I. only & 0.290 & 0.272 & 0.281 \\ % & 0.698 \\
\hspace{3mm}Biometrics + C.I. & 0.317 & 0.286 & 0.301 \\ % & 0.712\\
\hline
\textbf{Focus}\\
\hspace{3mm}Biometrics only & 0.251 & 0.256 & 0.253 \\ % & 0.716\\
\hspace{3mm}C.I. only & 0.332 & 0.342 & 0.337 \\ % & 0.742\\
\hspace{3mm}Biometrics + C.I. & 0.340 & 0.316 & 0.328 \\ % & 0.745\\
\hline
\end{tabular}
\caption{Comparison of the performance of predicting stress, focus and awakeness using the 3 different feature sets for the 10 participants. The performance is calculated as the average of the performance of the individual classifiers. Computer Interactions is abbreviated as C.I. here for readability. Precision and recall refer to the prediction of the more important classes, i.e. `stressed', `not awake', `not focused'.}
\label{ciPerformance}
\end{center}
\vspace*{-3mm}
\end{table}

As with the biometric models, the individual performance of both the computer interaction only models and the combined models varied quite a bit between participants. Using stress as an example again, in the computer interaction models 5 of the 10 participants saw improvements compared to the baseline, with a maximum improvement of 128\% in precision, and 78\% in recall. In the combined model for stress, 5 of the 10 participants saw improvements compared to the baseline, with a maximum improvement of 117\% in precision and 95\% in recall. Neither model was capable of correctly predicting any instances of 'stressed' for participant S7.

Since the number of features changes  depending on which feature set is used, we adjusted the feature selection parameter for each of the computer interaction and combined computer interaction/biometric models. The values reported in this section were achieved using the optimal feature selection parameters we found, which are shown in Table \ref{ciFeatureSelection}.

\begin{table}
\begin{center}
\begin{tabular}{lc}
\hline
Model/Feature Set & Number of Features Selected\\
\hline
\textbf{Stress}\\
\hspace{3mm}C.I. & 400\\
\hspace{3mm}Biometrics + C.I. & 800\\
\hline
\textbf{Focus}\\
\hspace{3mm}C.I. & 20\\
\hspace{3mm}Biometrics + C.I. & 300\\
\hline
\textbf{Awakeness}\\
\hspace{3mm}C.I. & All\\
\hspace{3mm}Biometrics + C.I. & 50\\
\hline
\end{tabular}
\caption{The optimal number of features we found to select for each of the model/feature set combinations. Computer interactions is abbreviated as C.I.}
\label{ciFeatureSelection}
\end{center}
\vspace*{-4mm}
\end{table}



\section{Analysis and Results}

\begin{table*}[h]
  \centering
  \includegraphics[width=1.0\textwidth]{rq1performance.pdf}
  \caption{Results of predictions using the individual models. The distribution columns show a bar chart of the response distribution (negative/positive) for each of the three variables. The baseline row represents the averaged results of our baesline classifier. The general row shows the averaged results of our models trained on all participants.}\label{tab:accuracy}%\todo{revise caption}}
  \vspace*{-4mm}
\end{table*}

To evaluate the efficacy of continuously predicting a knowledge worker's stress, focus, and awakeness in the workplace, we trained and tested machine learning classifiers using a leave-one-out cross validation. For prediction, we used the features extracted from the collected data as the input data, and binarized the participants' self-reported responses on stress, focus, and awakeness into two classes each (e.g., 'stressed' and 'not stressed') to use as output measures.

\subsection{Predicting Stress, Focus \& Awakeness}
% random forest best
\label{secOverallAccuracy}
As a first step, we compared multiple classifiers using the popular machine learning library scikit-learn~\cite{pedregosa11} and performing a grid search analysis to determine the optimal hyperparameters for each classifier. Our analysis showed that random forest outperforms all other classifiers, including Na\"ive Bayes, decision trees, support vector machine, random forest, and a multilayer perceptron neural network. The optimal values for random forest and the three output measures are listed in Table \ref{tab:hyperparams}. For the remainder of this paper, we will be referring to random forest classifiers trained with these hyperparameters.%\\[-0.1cm]
% for stress they were (minimum samples for split = 4, # of estimators = 100, max features = 0.5, k=300), for focus they were (minimum samples for split = 4, # of estimators = 50, max features = 0.5, k=200), and for awakeness they were (minimum samples for split = 4, # of estimators = 100, max features = 0.25, k=800)
\begin{table}[h]
	\begin{centering}
	\small\addtolength{\tabcolsep}{-1pt}
    \begin{tabular}{llll}
      \hline
      Variable & K & \# Estimators & Minimum Samples Split \\
      \hline
      Stress & 300 & 100 & 4\\
      Focus & 200 & 50 & 4\\
      Awakeness & 800 & 100 & 4\\
      \hline
    \end{tabular}
    \caption{The hyperparameters selected by grid search analysis to tune our random forest models. K refers to the number of features selected.}    \label{tab:hyperparams}
    \end{centering}
\end{table}



\noindent\textit{Results of Individual Classifiers}\\
Since peoples' experience of stress, focus, and awakeness (as well as their physiological manifestations) can vary a lot (e.g.,~\cite{Hernandez11}), we first trained individual classifiers for each participant (rather than a general one for all participants). The results of our analysis are reported in Table~\ref{tab:accuracy}. For our analysis, we report values of accuracy, one of the most commonly used metric to compare performance, as well as precision and recall of the classes of interest: 'stressed', 'not focused', and 'not awake'. Since the imbalance in the data can lead to high accuracy values if a classifier always just predicts the most likely/frequent class while ignoring the class of higher importance and interest, precision and recall of the class of interest can be more adequate metrics in this case~\cite{yap2014,bhattacharyya_data_2011,Hernandez11}. 
% However, the imbalance in the data can lead to a classifier with high accuracy if it always just predicts the more likely class while ignoring the class of higher importance and interest in our case (i.e. stressed, not focused, not awake). Therefore, we further report the precision and recall for the three classes 'stressed', 'not focused' and 'not awake'. 



Overall, we were able to use the extracted physiological features to predict all three aspects with reasonable accuracy, precision, and recall, as well as to improve upon the baseline---a stratified random classifier that randomly chooses one of the two classes with a bias towards the larger class. While the individually trained classifiers improved on average across all participants upon the baseline in all cases excepting recall of 'not focused', the improvement was substantially higher for awakeness (85\% improvement in precision,  62\% in recall, and 7\% in accuracy) than for stress or focus. Also, the performance of the individually trained classifiers varied greatly across participants. While some participants showed a large improvement, for others the baseline performed much better than the individually trained classifier. For instance, for predicting 'stressed', the individual classifiers improved upon the baseline for S4, S6, S8, S11, S12, and S14 with a maximum improvement of 88.4\% in precision and 184.6\% in recall for S12, while they did worse for S1, S3, S5, S7, S9, S10, and S13, and in the worst cases did not correctly predict a single instance of 'stressed'. Typically users that have the lowest precision and recall values are those where the data is the most unbalanced. This translates to a scenario where the classifier will likely label the minority instances as part of the majority class due to the unbalanced ratio of the data.\\[-0.1cm]
% We attribute these discrepancies to the large differences in response distributions between participants, as well as to the subjectivity of self-reporting.
%
%(improvement: 8\% precision, 1\% recall, 8\% accuracy), the individual performance varied greatly among participants. Some participants showed negligible difference in comparison to the baseline classifier, e.g. subject ..., while others showed a large improvement, e.g. subject ...
%\begin{table}[h!]
%\begin{centering}
%\begin{tabular}{lll}
%\hline
%Participant & Recall Improvement (\%) & Precision Improvement (\%)\\
%\hline
%S12 & 184.6 & 88.4 \\
%S6 & 66.7 & 85.8 \\
%S11 & 50.0 & 44.4 \\
%S8 & 32.4 & 26.7 \\
%S14 & 14.7 & 9.8 \\
%S4 & 5.6 & 5.5 \\
%S9 & -55.4 & -46.8 \\
%S3 & -90.0 & -82.1 \\
%S1 & -100.0 & -100.0 \\
%S5 & -100.0 & -100.0 \\
%S7 & -100.0 & -100.0 \\
%S10 & -100.0 & -100.0 \\
%S13 & -100.0 & -100.0\\
%S2 & - & -\\
%\hline
%\end{tabular}
%\caption{Percentage improvement in recall and precision for stress, using our approach compared to the baseline, on an individual level. For participant 2, the baseline achieved a precision and recall of 0, thus the improvement is undefined}
%\end{centering}
%\label{tab:indImprovement}
%\end{table}



\noindent\textit{Feature Selection and Importance}\\
There are a large variety of features that can be (and have been) calculated in previous research for each of the basic measurements listed in Table~\ref{signals}, such as the mean, standard deviation, maximum, and interquartile range. In addition, each of these metrics can be combined with the various time windows captured of a basic measurement, resulting in a large feature space. To reduce the feature space, we experimented with multiple feature selection methods, including selecting the top k highest correlated features by various metrics such as mutual information, Pearson's correlation coefficient, ANOVA's F-value, as well as wrapper methods such as recursive feature elimination, optimizing mean decrease accuracy by iteratively permuting features, and only selecting features that exceed a certain Gini importance threshold. We found that all methods produced similar results with respect to accuracy, precision, and recall for the individual models. Ultimately, we chose to use the top k features with the highest ANOVA F-value, as it is relatively simple and efficient to calculate. The values of k used were selected by grid search analysis, and are shown in Table \ref{tab:hyperparams}. 

%Respiration rate is also highly important for stress (#3 , 14.4%) but I'm not sure what previous works say about correlation with stress
Overall, the features that were selected as the important ones for the individual models based on the random forest algorithm varied greatly across participants. Yet, some feature categories were considered to be important more frequently than others. Table~\ref{tab:featureImportance} shows the averaged Gini importance for the feature categories used for predicting stress. Particularly important for stress were the feature categories heart rate variability (18.3\%) and skin temperature (15.7\%), both of which have been shown in several previous studies to be  indicators for stress~\cite{dishman2000stress,mcduff16,kataoka00}. For predicting focus, the feature categories for blood pulse wave (14.2\%) and heart rate variability (13\%) showed to be the most important categories, while for awakeness the most important ones were heart rate variability (13.6\%) and blood pulse wave (13.1\%).\\[-0.1cm]


\begin{table}[h!]
  \begin{centering}
  \begin{tabular}{llll}
    \hline
    Feature Category & Stress & Focus & Awakeness\\
    \hline
    Heart Rate Variability & 18.3\% & 13\% & 13.6\%\\
    Blood Pulse Wave & 10\% & 14.2\% & 13.1\%\\
    Heart Rate & 8.7\% & 12.6\% & 10.3\%\\
    Skin Temperature & 15.7\% & 9.8\% & 10\%\\
    Galvanic Skin Response & 6.6\% & 8.2\% & 5.1\%\\
    Respiration Rate & 14.8\% & 12.7\% & 10\%\\
    Oxygen Saturation & 5.6\% & 4.3\% & 2\%\\ 
    Energy Expenditure & 6\% & 7.7\% & 4.8\%\\
    Activity & 4.6\% & 7.8\% & 7.6\%\\
    Steps & 1.7\% & 0.8\% & 0.9\%\\
    Time of Day & 0\% & 0.1\% & 0.5\%\\
    \hline
  \end{tabular}
  \caption{The averaged Gini importance of each feature category, per response variable.}
  \label{tab:featureImportance}
  \end{centering}
  \vspace*{-2mm}
\end{table}


\noindent\textit{Individual vs.\ General Model}\\
Individual models are trained specifically for each individual and thus require a data collection period before they are capable of making accurate predictions. On the other hand, the idea of general models is to be able to train them on already collected data and then  to be able to apply them even to new and unseen individuals, thus overcoming the cold-start problem. Given the big individual differences in biometrics, training a general model to achieve an adequate accuracy for new individuals is not necessarily possible. 

To examine the performance of a general model for our participants, we trained three general models, one for focus, one for awakeness and one for stress. We roughly followed the same procedure as for the individual models. Due to the larger amount of data available in the general case, we used the more common random undersampling, which randomly selects elements in the majority class to exclude from the dataset, instead of random oversampling to balance the distribution of the dataset. The models were trained on the datasets of 13 of the 14 participants, and then evaluated on the dataset of the last, repeating this process for all 14 participants. 

The bottom row of Table \ref{tab:accuracy} presents the averaged performance results for this approach in terms of accuracy, precision, and recall. Although the averaged precision and recall are comparable or better than those of the averaged individual results, this was at the cost of a large decrease in overall accuracy. Upon closer investigation into the performance of the general model when testing on each participant, we found that individually trained models for each participant performed much better than a general model trained over all participants. Using stress as an example, for participant S12, for whom we saw the greatest increase compared to the baseline in individual models, the general model was unable to predict a single instance of 'stressed' correctly. This is consistent with our expectations because biometric features are highly specific to individuals.


%\todo{put the content of the following sentence somewhere into the discussion; We attribute these discrepancies to the large differences in response distributions between participants, as well as to the subjectivity of self-reporting.}


%%!TEX root = bioPrediction_main.tex

\subsection{Minimum Number of Training Samples}\label{secLearningCurve}
Collecting experience samples from users is expensive, since participants 
are being interrupted several times a day and have to answer the survey. 
To minimize the number of samples to be collected from participants, we 
examined how the performance of individual classifiers changes over the 
number of samples used to train the classifier.

Participants in our study had varying levels of responsiveness to the 
experience sampling ranging from 10 to 76 (see Figure~
\ref{responseDistribution}).
%Answering Reviewer 2's concern: Why only 10 participants? What was the rationale for the cut-off selected? 
To answer this research question we analyze the maximum number of responses \textbf{all} analyzed participants have to train the model. Since some participants have a very small number of maximum answers (ten being the lowest) we excluded the four participants with the least number of responses for this analysis.
%To maintain a certain generalizability while 
%also being able to examine a range of sample numbers for training the 
%classifier, we removed the four participants with the least number of 
%samples for this analysis. 
As a result, we obtained a corpus of analysis where all participants have a 
higher number of responses (34 being the lowest), which allows us to examine 
the learning curve of the classifiers for a larger number of training data points. For our 
analysis, we thus performed a leave-one-out cross validation with random 
sample sets of size 1 to 33 and calculated the average through all folds of 
the validation.

For each of the three productivity-related aspects, we are again more 
interested in predicting when a worker is stressed, not awake, or not 
focused, the less common class in all three cases. Since the less common 
class can be very small, we weighted each participants' classifier 
performance by the percentage of the samples in this smaller class.


\begin{figure}
  \centering
          \includegraphics[width=0.5\textwidth]{20180912AwakenessLC2.png}
  \caption{Performance of the per participant trained awakeness classifiers, measured in precision, recall (sensitivity), and F-score. The dotted red line represents the F-score trend.}\label{fig:learningCurveInd}
  \vspace*{-3mm}
\end{figure}

\vspace{0.05in}
\noindent\textit{Individual Classifiers for Binary Prediction}\\
The averaged performance of all individually trained random forest classifiers for 'awake' with respect to the training sample size is presented in Figure~\ref{fig:learningCurveInd}. The trend indicates a positive correlation between the number of samples in the training set and the classifiers' F-score performance, with an overall improvement of  114\% (from 0.14 to 0.30) in the F-score between a training set of one sample to one with 33 samples. The trends for
the remaining indicators are 29\% for stress and no overall improvement for focus.\\[-0.1cm]

%\noindent\textit{}
%When contrasting
%models trained and tested on the 
%biometric data of each user individually, against
%a general model trained and tested with the data of all users. 
%We can conclude
%that for up to 33 training samples analyzed in this study,
%the former has better performance when predicting the
%responses of each user than when training 
%a model with the data of several different users. 
%This is partially due to the 
%variability across users and the subjective nature
%of the responses (e.g.,where slightly awake for one person
%may be not awake for another). 

\begin{figure}
  \centering
      \includegraphics[width=0.5\textwidth]{20180914Awakeness5PointScaleOnly15Lines.png}
  \caption{Performance of the per participant trained classifiers for predicting 5-point awakeness, measured in root mean squared error.}
   \label{fig:learningCurve5}
\end{figure}

\noindent\textit{Predicting Five Classes}\\
In a second step, we analyzed a more fine-grained prediction using the initial 5-point Likert scale responses rather than the binarized ones as output measure. Figure~\ref{fig:learningCurve5} depicts the performance of the individually trained classifiers in terms of the root mean squared error. The root mean squared error represents the distance of the predicted from the actual value, which provides a more nuanced measure of the performance in the fine-grained prediction case. The figure shows a similar trend as for the binarized prediction, in that the root mean square error averaged over all ten participants decreases with more samples (from 0.49 to 0.41 root mean square error) and thus the performance increases. At the same time, the figure also shows that the performance results for the fine-grained prediction, again, vary substantially across participants.

\subsection{Minimum Time Window}\label{secMinimumTW}
In general, the less biometric data is needed to accurately predict a certain outcome measure, the easier and faster the analysis and data collection. To examine the optimal and minimum time window for the prediction of stress, focus, and awakeness, we used 16 different time windows from 10 seconds to 3 hours as depicted in Figure~\ref{timeWindows}. For our analysis, we then trained individual classifiers for each of the 16 time windows, using only features that had a time window smaller or equal to the time window rather than using all combinations of $\{Biometric Measures\} X \{Statistical Metrics\} X \{Time Windows\}$. We again used random forest and a leave-one-out cross validation to train individual classifiers. Since the number of features used for the training changed with each time window, we did not apply our feature selection in this case, but used all features available. Finally, due to the imbalance in the data, we again weighted each participants' classifier performance by the number of instances in the smaller class to calculate the average.

\begin{figure}
  \centering
      \includegraphics[width=0.4\textwidth]{timeWindows.png}
  \caption{Time windows of biometric data collected prior to each survey response: 10sec, 20sec, 30sec, 45sec, 1min, 2min, 3min, 5min, 7.5min, 10min, 20min, 30min, 45min, 1hour, 2hour, and 3hour.}
   \label{timeWindows}
\end{figure}



\begin{figure}
  \centering
      \includegraphics[width=0.5\textwidth]{20180912AwakenessTWBars.png}
  \caption{Performance (F-score) of individual classifiers trained on the different time windows to predict `awake'.}
  \label{timeWindowsPandR}
  \vspace*{-3mm}
\end{figure}

Figure~\ref{timeWindowsPandR} shows how the F-score changes for predicting 
`awake' over the 16 different time windows. The figure shows an increasing 
trend in the F-score, i.e. the higher the number of included time windows, 
the higher the F-score. However, there is one exception, the time window of 
1200 seconds that achieves a performance close to the one for the time 
window 10,800 seconds (3 hours), at which point all features are included. 
Overall, our results thus show that while using all time windows up to 3 
hours performs best, and outperforms the feature set that is solely based on 
a 10 second time window by 28\% (from 0.18 to 0.24), the performance for a 
time window of 1200 seconds is a good trade-off for selecting a shorter time 
window while maintaining high performance.



%We also compared the performance of a model created per user against
%a model created across users.
%To calculate the performance of the model created per user,
%for each user we create a model using the features 
%related to each time window for the selected user only, 
%while in the model created with data across users,
%we trained the model with the data related to each
%particular time window of all users.
%our results show that the personalized model created
%for each user outperforms in every case
%the model created the with data across all users.

%This study
%shows that in general larger time windows predict stress,
%focus, and awakeness more precisely than 
%shorter time windows. We can also notice
%a very gentle upwards trend
%with a inflexion points at 45 seconds and 180.
%There is a peak of performance in 45 seconds, 
%time window that could be used if time is a scarce resource.
%The performance stabilizes around 180 time window
%which indicates that collecting data for a longer time
%period will not increase significantly the performance of the approach.
%The difference in performance between the 
%lowest performing and the highest performing time window is 
%less than a 10\% improvement. 


\subsection{Computer Interaction Data} \label{secCI}
Given our focus on knowledge workers (i.e., workers who generally spend a lot of time interacting with information on their computer at work), we also analyzed the use of computer interaction features to predict focus, awakeness and stress. To collect computer interaction data, we used an open source computer interaction monitor (reference omitted for double-blind.)
%the open source PersonalAnalytics project\footnote{https://github.com/sealuzh/PersonalAnalytics}  \cite{meyer18} 
to track participant's mouse and keyboard activity, as well as details about their active window. The specifics of the features tracked are listed in Table \ref{tracker}. The tracker was installed on the computers of 10 of the 14 participants, with participants S6, S10, S12, and S13 opting out of this part of the study due to privacy concerns.  Therefore, we limited this analysis to the 10 participants for whom we could calculate all features. 

\begin{table}
\begin{center}
\small\addtolength{\tabcolsep}{-1pt}
\begin{tabular}{l l}
\hline

Feature collected by tool & Description \\ 
\hline
Total keystrokes per min& Sum of all types of keystrokes \\ 
Normal keystrokes per min&F[h] Not backspace and navigation \\ 
Backspace keystrokes per min& Backspace keystrokes \\ 
Navigation keystrokes per min& Arrow key keystrokes \\ 
Total clicks per min& Sum of all click types \\ 
Other clicks per min& Not right and left clicks \\ 
Left clicks per min& Left clicks \\ 
Right clicks per min& Right clicks \\ 
Scrolled distance per min& Scrolled distance in pixels \\ 
Moved distance per min& Mouse movements in pixels \\ 
Activity switches per min& Browser window title changes \\ 
Category switches per min& Activity performed category \\ 
\hline
\end{tabular}
\caption{Features collected per user by the computer interaction tracker}%~\cite{meyer18}}
\label{tracker}
\end{center}
\vspace*{-1mm}
\end{table}

For calculating computer interaction features, we again used the aforementioned 16 time windows and scaled the computer interaction values if the time windows did not align. For our comparative analysis of the different sensing techniques---biometrics vs computer interaction---we then created two new feature sets for each participant in addition to the biometric one: one with only computer interaction features, and one with computer interaction features plus biometric features. 

Table~\ref{ciPerformance} lists the results of our analysis. The results show that in all cases, the computer interaction based model was able to improve upon the biometric model in terms of precision and recall.
%MS is removing the accuracy results because it is not what we care about and it is just making the results harder to understand
%, but not in accuracy. 
Further, we found that the combined model was the most effective model in terms of precision and recall for predicting stress and awakeness overall, but performed slightly worse than the model using only computer interaction features for focus. 
%Since we consider precision and recall for predicting the class of higher interest, i.e. stressed, not focused, not awake, to be the most important statistics when interpreting our results, we compared the models against each other by averaging these two statistics.


\begin{table}
\begin{center}
%\small\addtolength{\tabcolsep}{-1pt}
\begin{tabular}{llllll}
\hline
Model/Feature Set & Precision & Recall & F-Score \\ %& Accuracy\\
\hline
\textbf{Awakeness}\\
\hspace{3mm}Biometrics only  & 0.269 & 0.314 & 0.289 \\ %& 0.808\\
\hspace{3mm}C.I. only  & 0.425 & 0.362 & 0.391 \\ % & 0.758\\
\hspace{3mm}Biometrics + C.I. & 0.390 & 0.404 & 0.400 \\ % & 0.791 \\
\hline
\textbf{Stress}\\
\hspace{3mm}Biometrics only  & 0.270 &	0.260 & 0.265 \\ % & 0.775\\
\hspace{3mm}C.I. only & 0.290 & 0.272 & 0.281 \\ % & 0.698 \\
\hspace{3mm}Biometrics + C.I. & 0.317 & 0.286 & 0.301 \\ % & 0.712\\
\hline
\textbf{Focus}\\
\hspace{3mm}Biometrics only & 0.251 & 0.256 & 0.253 \\ % & 0.716\\
\hspace{3mm}C.I. only & 0.332 & 0.342 & 0.337 \\ % & 0.742\\
\hspace{3mm}Biometrics + C.I. & 0.340 & 0.316 & 0.328 \\ % & 0.745\\
\hline
\end{tabular}
\caption{Comparison of the performance of predicting stress, focus and awakeness using the 3 different feature sets for the 10 participants. The performance is calculated as the average of the performance of the individual classifiers. Computer Interactions is abbreviated as C.I. here for readability. Precision and recall refer to the prediction of the more important classes, i.e. `stressed', `not awake', `not focused'.}
\label{ciPerformance}
\end{center}
\vspace*{-3mm}
\end{table}

As with the biometric models, the individual performance of both the computer interaction only models and the combined models varied quite a bit between participants. Using stress as an example again, in the computer interaction models 5 of the 10 participants saw improvements compared to the baseline, with a maximum improvement of 128\% in precision, and 78\% in recall. In the combined model for stress, 5 of the 10 participants saw improvements compared to the baseline, with a maximum improvement of 117\% in precision and 95\% in recall. Neither model was capable of correctly predicting any instances of 'stressed' for participant S7.

Since the number of features changes  depending on which feature set is used, we adjusted the feature selection parameter for each of the computer interaction and combined computer interaction/biometric models. The values reported in this section were achieved using the optimal feature selection parameters we found, which are shown in Table \ref{ciFeatureSelection}.

\begin{table}
\begin{center}
\begin{tabular}{lc}
\hline
Model/Feature Set & Number of Features Selected\\
\hline
\textbf{Stress}\\
\hspace{3mm}C.I. & 400\\
\hspace{3mm}Biometrics + C.I. & 800\\
\hline
\textbf{Focus}\\
\hspace{3mm}C.I. & 20\\
\hspace{3mm}Biometrics + C.I. & 300\\
\hline
\textbf{Awakeness}\\
\hspace{3mm}C.I. & All\\
\hspace{3mm}Biometrics + C.I. & 50\\
\hline
\end{tabular}
\caption{The optimal number of features we found to select for each of the model/feature set combinations. Computer interactions is abbreviated as C.I.}
\label{ciFeatureSelection}
\end{center}
\vspace*{-4mm}
\end{table}


%!TEX root = bioPrediction_main.tex
\section{Discussion}\label{sec:Discussion}

Humans experience stress, focus and awakeness in different ways.  In
this paper, we have attempted to study these mental states in the
workplace using both participant self-reports and biometric data. We
discuss implications from our study for the workplace, including ways
in which the information might inform digitally-controlled or
digitally-informed parts of the workplace. We also discuss 
challenges imposed by the data and possible future paths of
research.

\subsection{Implications for Workplaces}
Being able to accurately recognize periods of high-stress in knowledge
workers could enable more respectful workplaces. For example, an
ability to sense and predict stress in the moment could help companies
to prevent or de-escalate potentially dangerous situations in the
workplace, e.g. confrontation between co-workers. Building an 
understanding of stress and focus over time and in the moment
could also help create workplaces that are more conducive
to enabling knowledge workers to be more productive. For example,
this information could be used to feed an awareness dashboard of a team's stress level,
and avoid digital interruptions in high-stress or high-focus periods
similar to previous studies~\cite{zuger2017reducing}.  Building
an understanding of awakeness and focus could also enable the creation
of workplaces that are conducive to workers producing higher-quality
work. For example, if awakeness or focus decreases, they might 
be enhanced by adapting lighting in the workplace 
or scheduling breaks to prevent focus loss.

Currently, the cost of biometric sensors and necessary infrastructure,
such as automated light and sound systems for adjusting the
environment, makes our approach most appropriate for high-value
workspaces, such as control rooms, command centers, or dispatch
offices. However, as standard office settings become more
personalizable (e.g., via adjustable desks, lighting, and sound
showers) and sensor costs decrease, our approach could be applied to
any office environment, and thus could impact a large percentage of
modern workers. As in modern cars, temperature and lighting could be
regulated on a per-person basis, which would allow the environment to
react to the person's current state and to maximize each person's
preferences and productivity (e.g., preferences of men and women in
temperature~\cite{Karjalainen07}). 
We believe that our results present a good step to more in situ usage of biometric sensing. Future studies can build on the evidence, and for example, reduce the effort required for the experience sampling from continuously rating stress levels to using biometrics as a ground truth with occasional validation of the predicted stress levels, or
identify how to balance needs across a group of office workers and how
to handle conflicting levels between different group members.



%%\vspace{0.05in}
%\noindent
%\textbf{Implications for HCI Design and Workspaces}
%% The results discussed in this paper establish great potential
%% to increase the well-being of knowledge workers and the quality of their work
%% through monitoring their biometric signals and computer interactions with the
%% goal of recognizing their stress, awakeness, and focus levels in an uninvasive 
%% and automatic manner. e

\subsection{The Effect of Tai Chi on Stress}
As described in our study procedure,
due to our focus on stress, we offered participants an opportunity to be part of Tai Chi classes.
The two primary reasons for this intervention was to keep participants motivated to continue the self-reports over the long study period, and to offer a technique that might help reduce stress.
After consulting with other researchers, we decided to offer Tai Chi in the last month of the study, leaving the first month of the study unaltered. While participants were not required to take a Tai Chi class, they all took one per week except for two participants that opted out of the last two weeks of classes. Four participants explicitly stated that they liked the Tai Chi sessions or that they were ``excellent''.
These classes were performed once per week with a duration of one hour.

While this was not the focus of our study, we performed a secondary analysis to examine whether the Tai Chi classes had any effect on the participants' stress levels in the work days directly following the Tai Chi session. For this, we build a linear mixed model with the self-reported daily stress level as dependent variables and the participants as random effects. We found that Tai Chi attendance contributed a small amount to decreased stress in the work week immediately following the Tai Chi session (slope of -0.188, p <0.05). However, we did not find a connection between attendance and stress on the day of the session suggesting that Tai Chi might have long-term effects, but is not conducive at relieving stress close in time to the intervention. Overall, the Tai Chi thus had a small impact on the collected data of the second month of the study, which poses a threat to the validity to our observed trends for this period of time. However, since knowledge workers might attend these kinds of classes on their own doing, we believe that this is negligible, and the analysis rather provides a weak indication that this kind of stress intervention might in fact help to reduce stress in the wild.

%% %\vspace{0.05in}
%% \noindent
%% \textbf{Ground Truth and Self-Reporting:} 
\subsection{Ground Truth and Self-Reporting}
Studying mental states, such as stress, awakeness and focus,
requires collecting a valid ground truth from each participant.
We spent considerable time when designing the study
 determining the exact questions to ask of participants,
consulting experts in the area, and basing
the questions and wording on previous research and studies. 
Despite the care taken, it is possible that the 
gathered data lacks reliability and validity. Some have
questioned the
reliability and validity of self-reports as we used in our study
due to subjective biases, lack of care in reporting, and the
highly individual nature of reporting aspects such as
stress~\cite{Hernandez11,Hovsepian15}. 

In addition, in contexts such
as the workplace, as in our study,
participants  might be afraid to genuinely report levels
of aspects, such as sleepiness. Hence, there is a chance that the
self-reports we gathered  do not adequately reflect the ground
truth of the underlying variables under investigation. It could even be
the case that certain biometrics might represent a more accurate
ground truth of the studied phenomena than the self-reports. This
suggests that a more confirmative study rather than an inquiry study
could be a better approach, and we will explore such routes in future
work.

%% %\vspace{0.05in}
%% \noindent
%% \textbf{Imbalanced Data:} 
\subsection{Imbalanced Data}
Study participants provided highly
imbalanced data in their survey responses, with most participants only
taking advantage of a subset of the Likert-scale values and the data
points mostly being clustered around the middle of the scale, as can
be seen in Table~\ref{responseDistribution}. While some of the
imbalance is expected due to certain classes, such as 'not stressed',
being more common in the workplace, this imbalance also provides
challenges in the training and assessment of a machine learning
classifier, as also found by others, e.g.~\cite{Exler16}. We addressed
this for the training by oversampling in case of few data samples for
the individual models and undersampling in case of a general model
where more data was available. \rev{Rebalancing the dataset using such techniques
is a common and effective practice~\cite{branco2016survey}. Alternative techniques such as SMOTE~\cite{chawla2002smote} exist and can perform better than those we employed, however they are impractical considering the limited amount of data we have to work with. Oversampling the dataset as we did may also lead to an increased risk of model overfitting. However, we believe the benefits of rebalancing the dataset outweigh this possible downside.}

Especially in light of the imbalance
in the data, the results we achieved with our models are
encouraging. For the assessment of the classifiers' performance we
addressed the imbalance by not just presenting accuracy, but also by
focusing on prediction and recall to examine the classifier's
performance in predicting the infrequent (yet more important) cases,
such as when a user is struggling to stay awake and an intervention or
warning might be needed most.
% \textbf{Unbalanced Data:}
% As shown in Table~\ref{responseDistribution}, the typical participant provided highly unbalanced survey responses, with most responses exhibiting a central tendency. Many participants provided zero or one very high (or very low) value on the Likert scale. In spite of this, our results show that by using biometric sensors and leveraging a machine learning approach, we are able to predict stress, focus, and awakeness better than the baseline, with improvements up to 84.64\%. Taking into account that the model correctly predicts many cases that happen infrequently, this is especially promising. These infrequent cases, such as when a user is highly stressed or struggling to stay awake, are in fact the most important cases to detect, as they are when intervention is needed most.

\subsection{Predicting Stress with Computer Interaction Data}
Knowledge workers, a focus of our work and study, often spend
a large amount of time each day interacting with information on 
their computer at work. An interesting direction for future study
is to consider whether this computer interaction data, which can
be gathered non-invasively as work occurs, could serve to sense
and predict stress, focus and awakeness. Features that could be
investigated include keystrokes per minute, mouse clicks per minute
and changes in the active window title.




%% %\vspace{0.05in}
%% \noindent
%% \textbf{Corpus Size:}
%% In this study we collected data from 14 professionals over the course of eight weeks. By scholarly standards this constitutes a large dataset --- especially given that the participants were not students. However, in the context of machine learning this corpus is relatively small. Thus, the model's overall performance, while promising, is only an indication of the performance that a larger corpus could provide. Our results show that there is a positive correlation between data added to the training corpus and the performance level of the model. This upwards trend likely continues, but a larger dataset is needed to confirm or reject this hypothesis. 

% \vspace{0.05in}
% \noindent
% \textbf{Data Collection:} 
% One advantage of our study is that it was conducted on professionals in a real office setting (i.e., not on students). However, the disadvantage to working with professionals is that business deadlines and other pressing real world issues lead to missing survey responses. Additionally, busy professionals often fall into the trap of always selecting the same value for a given question that they have answered many times, which reduces the accuracy of the collected data. To minimize the effect of this inertia, we advise collecting data at typical low-productivity times, such as in the early afternoon or on Mondays~\cite{mark2014bored}.

%% %\vspace{0.05in}
%% \noindent
%% \textbf{Per-User Model vs. Across-Users Model:}
%% Our results show that individually trained models outperform on average a more general model. Yet, given the increase in model performance with training data set size, we expect that with enough data, a generic model will perform reasonably close to the individually trained models. This would have significant practical implications, eliminating the need for survey-based feedback (which is only necessary for model training), and thereby completely automating measurement of human aspects. Our follow up work will focus on gathering more data toward this end.


% \vspace{0.05in}
% \noindent
% \textbf{Per-User Model vs. Across-Users Model:}\
% While we do see variations across participants, and thus our work shows that personal 
% models perform better than generic models, we expect that with enough data, a generic 
% model may perform equally well. This would have significant practical implications, eliminating the need for survey-based feedback (which is only necessary for model training), thereby completely automating measurement of human aspects. Our follow up work will focus on gathering more data toward this end and on evaluating the performance of a generic model.



%!TEX root = bioPrediction_main.tex
\section{Threats to Validity}
There are numerous threats to validity to our study.

\textbf{External Validity:}
It is a threat that the results of this study
will not generalize to a broader population of office workers.
To address this concern, we collected participants
from a wide variety of different departments
with different age ranges, genders, work experience, and 
working in different positions.

Another threat is that our results might not generalize
to a different office environment. We have conducted
this study in a typical office environment, common
among technology workers across the world.
These office environments control for a series of
variables to make them standard world wide such
as controlled temperature and lighting.

\textbf{Internal Validity:}
This study tries to find correlations between
biometric features and different productivity-related indicators (stress, focus, and awakeness).
Nonetheless, biometric signals are influenced by far more
variables than the ones this study comprehends.
Therefore, trying to draw a strong causality between the biometric
features and the productivity-related aspects would be inaccurate.
To address this concern we collected the data
in a rote environment and in a regular manner 
to minimize the number of 
external causes that may affect each participant's
biometric signals.

%It is possible that the amount of data collected
%is not enough
%to draw valid conclusions. To address this threat, 
%we collected a data for an eight week period, which is
%400\% longer than previous studies~\cite{zuger18,Muller16}.


\textbf{Construct Validity:}
A threat to the study is that
there are other factors that might either influence the
human aspects of interest or that were considered but
are unrelated biometric signals.
To mitigate this threat we used an state-of-the-art
device that captures a large number of highly accurate biometric
signals. We collected the most commonly analyzed
biometrics that historically have shown correlation with 
productivity aspects from each user.
We also perform a grid analysis in our machine learning model
to pick the number of features that is more predictive
of the productivity-related indicators, 
therefore removing the biometric
features that may be unrelated to the indicators.












%!TEX root = bioPrediction_main.tex
\section{Conclusions} 

\vspace{-4mm}
Stress, awakeness, and focus at work are highly relevant aspects when
it comes to productivity and well-being at the workplace. In this
paper, we presented the results of a study with 14 professional
knowledge workers in their workplace over an eight week period to
better understand how workers experience stress over time and to
examine the ability of biometrics to predict these
productivity-related aspects. The longitudinal and in-situ placement of
the study support and extend previous work. Based on
twice collected daily survey responses, we observed that
although participants sometimes saw periods of sustained stress,
they would always return to a baseline stress level for them
at some point. We also observed that stress levels seldom spiked,
but when they did rise, the rise in stress tended to last more
than a day. In addition to the survey responses, we 
continually collected
biometric data with which we were able to create a model that is able to
predict user stress, sleepyness, and lack of focus with an up to 84.65\% improvement over the baseline.

These results open up new opportunities to help increase knowledge
workers' productivity and well-being, ranging from instantaneously
taking action to prevent potentially risky situations and prevent
accidents due to a lack of focus or awakeness, all the way to
recommending interventions to reduce stress if it becomes more
chronic.

% \todo{as you can see, the conclusion is way too long and has some cut and paste repetition}

% Stress, awakeness and focus at work are highly relevant aspects when it comes to productivity and well-being at the workplace. In this paper, we presented the results of a study with 14 professional knowledge workers in their workplace over an eight week period to examine the ability of biometrics for predicting these producitivity-related aspects. The longitude and in situ placement of the study as well as the breadth of human aspects examined, support and extend previous work. Based on twice collected daily survey responses and continually collected biometrics data we were able to create a model that was able to predict , outperforming the baseline by as much as 84.6\%, in the case of awakeness. We also show that there is a positive correlation between number of training instances and model performance--a 114\% improvement from training with one sample to training with 33 samples--suggesting that with a larger amount of training data prediction would continue to improve. Due to this, while current user-specific models have higher performance than across-user models, there is hope that with enough data even across-user models could become accurate. 


% Our analysis shows that we are able to \todo{fill this}

% Our study with 14 professional knowledge workers in their workplaces over an eight week period has shown that we can use biometrics to predict these producitivity-related aspects of an individual knowledge worker. The longitude and in situ placement of the study as well as the set of human aspects examined support and extend previous work. In addition, our analysis 

% as well as computer interaction features to predict these aspects of an individual knowledge worker
% has shown that we can use biometrics as well as computer interaction features to predict these aspects of an individual knowledge worker over a long period of time in situ
% The results also show that the general ... \todo{fill this}

% In this study we analyze biometric signals of office workers to predict the productivity-related human aspects of stress, focus, and awakeness. A key component of this study was its length, an eight week period, longer than previous studies by a factor of four~\cite{zuger18,Muller16}, and its realism , studying professionals in situ instead of students. We collected information from 14 participants from different departments within the company, covering different ages ranges, genders, and work experience. By learning from twice daily survey responses and continually collected biometrics data we were able to create a model that was able to predict human aspect values, outperforming the baseline by as much as 84.6\%, in the case of awakeness. We also show that there is a positive correlation between number of training instances and model performance--a 114\% improvement from training with one sample to training with 33 samples--suggesting that with a larger amount of training data prediction would continue to improve. Due to this, while current user-specific models have higher performance than across-user models, there is hope that with enough data even across-user models could become accurate. 

% From these results emerge a series of opportunities to help increase awakeness, focus, and reduce stress in office workers. In the near future, we will therefore be able to detect a decrease in productivity-related aspects following our approach, and be able to take action to prevent potentially risky situations.

%\begin{acks}
%The acknowledgments section will be included for the camera-ready version of the paper.
%\end{acks}



\balance{}

% REFERENCES FORMAT
% References must be the same font size as other body text.
%\bibliographystyle{ACM-Reference-Format}
\bibliography{bp_bibliography}

\end{document}\endinput
